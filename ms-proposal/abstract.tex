%!TeX root = SM_Thesis_Proposal.tex

\begin{abstract}

    Linux is an operating system that underlies most of modern computing
    infrastructure. As such, it has conflicting requirements of being both
    general purpose and performant. This thesis identifies a limitation in
    Linux's interface for a use case that is both generic enough and important
    enough to deserve first class support in the kernel: the separation of
    latency critical (LC) workloads from best effort (BE) ones. The current
    interface for separating these two, used by popular containerization
    software such as Kubernetes and AFaaS, is based on weights and is poorly
    enforced. This thesis explores making the LC/BE split explicit by putting
    best effort work in a separate scheduling class. We show that doing so
    enables Linux to enforce the LC/BE split across cores with acceptable
    overheads, and that this dramatically improves the performance of LC tasks
    in the presence of BE workloads.

\end{abstract}

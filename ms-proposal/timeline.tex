%!TeX root = SM_Thesis_Proposal.tex

\section{Timeline}
\label{sec:timeline}

\textbf{1. Current state of the project:} 
The driving problem that motivated the project was multiple experiences of
confusing performance results in other projects, which pointed to a gap between
expectation and reality of Linux's cgroup interface. 

We have done a comprehensive analysis of the problem at different levels, by
looking at Linux's scheduling code as well as running many benchmarks and
profiling traces. This has allowed us to narrow down the problem to a simple
case that exhibits the undesirable behavior, and to be able to point to what
goes wrong, both in the traces and in the code.

We have experimented with different solution approaches, and landed on one that
requires editing the Linux migration code during scheduling decisions. We are
now at a point where we have a patch that leads to the desired behavior, and
have shown that it solves the simple case that we initially narrowed the problem
down to.

\textbf{2. Steps remaining:} In the time between now and the deadline, we plan
to further understand and push the solution. This includes three main
dimensions: we want to measure the exact costs (how long does the added code
take to run); we want to understand its implications (do things break when BE
processes are starved for longer periods of time); and finally we want to create
more realistic and end-to-end scenarios to understand the practical implications
of the change we made.

\textbf{2.a Measure the cost:} We plan to run a small microbenchmark where we
trigger the new patch code by running three \schednormal{} tasks and one
\schedidle{} task on two cores. We know the kernel will, once the \schednormal{}
task that has a core to itself completes, steal one of the others. We can use
that opportunity to measure how long it takes to run the code in the worst case
scenario, ie when it has to check the load and then also actually go and steal work.

\textbf{2.b Understand the implications:} Linux has many mechanisms to avoid
starvation of processes, and the change we make allows for starvation. We will
test this by starving a complex BE task (ie one that has open TCP connections,
files, etc) for a long period of time and ensuring that it can still continue
running as normal after.

\textbf{2.c End to end scenario:} The initial problem came from more complex
examples of more realistic applications than what we ended up reducing it to, so
we want to take the solution and apply it to these settings and demonstrate that
it improves performance as expected.



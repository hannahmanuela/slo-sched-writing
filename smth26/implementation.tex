\section{Implementation}\label{s:implementation}

A categorical differentiation in the scheduler allows for fewer synchronization
points without compromising the enforcement of global isolation. In order to
enforce a categorical priority, the scheduler only needs to sychronize at two
points: on \textit{entry}, when a new high-priority thread wakes up on a core
already running something high-priority, and \textit{exit}, when starting to run
a low priority thread. If on entry the scheduler looks for cores running low
priority work to go interrupt, and on exit the scheduler looks for high priority
work to steal, it has ensured the global property that \textit{no core is ever
running a BE process while an LC one is runnable and waiting}.

In order to affect the changes required, we modify Linux and put in the pieces
to make \schedidle{} effectively its own scheduling class. We chose to do this
rather than create an entirely new scheduling class in the code, which would be
intellectually cleaner and closer to the conceptual reality, but would require
more code changes and leave \schedidle{} in it's current half-formed
state.

We patch Linux version 6.14.2, and the patch requires XX lins of code change. It
intervenes in two places in the current scheduling process of the Normal
scheduler: 
\begin{enumerate}
    \item it ensures that no \schedidle{} entity will be chosen if there is a
runnable \schednormal{} entity (this overrides the fair share that even a weight
1 process would occasionally get in the unmodified kernel),
    \item it tries to steal queued \schednormal{} entities from other cores
before running a \schedidle{} entity.

\end{enumerate}


Doing the first part required changing the code that picks the next entity in
the Normal scheduler, in the function \texttt{pick\_eevdf}. The Normal scheduler
maintains the runqueue in the structure of a binary tree, which is ordered by
the amount of virtual CPU time each process has gotten (virtual because it takes
into account the entities weight). The current function searches that tree, and
chooses the most leftmost eligible entity to run. The patch first adds checks
whether the runqueue is currently in an `idle' state, ie only has \schedidle{}
entites on its runqueue. If that is true then the search is performed as usual,
if not then the patch checks each movement through the tree to ensure that the
subtree being chosen has at least one \schednormal{} entity, and that the final
process chosen is one as well. This ensures that the most eligible
\schednormal{} task is chosen to run next.

Should the task chosen be in \schedidle{}, but the one running before was
\schednormal{}, then we know we are in an \textit{exit} path. The patch adds
code that, in this situation, will try to steal queued \schednormal{} tasks from
other cores. It will steal up to half of the busiest cores' queued
\schednormal{} tasks.


\section{Implementing \beclass{} in Linux}\label{s:implementation}

In order to enforce latency critical service's CPU reservations, the goal of the
implementation is to enforce the maxim that no \beclass{} userspace process is
ever running on cores reserved for a \normalclass{} workload if a \normalclass{}
process is runnable and queued. In order to do so, the scheduler needs to
enforce the priorities in three different places:
\begin{enumerate}
    \item \local: in picking the next process to run on each core, it ensures
        that no \beclass{} process will be chosen if there is a runnable
        \normalclass{} process,
    \item \entry: when waking up a \normalclass{} process on a core already
        running a \normalclass{} process, it looks for other cores running
        \beclass{} entities to go interrupt,
    \item \exit: when the last \normalclass{} process on a core's queue blocks
        or exits, the core tries to steal queued \normalclass{} processes from
        other cores before running a \beclass{} process.
\end{enumerate}

In order to implement \beclass{} in Linux, we build on \schedidle{}, an existing
scheduling \textit{policy} in Linux, rather than creating an actual new
scheduling class, because \schedidle{} already has some of the features we want
for \beclass{}.

\subsection{\schedidle{} lends itself well to \beclass{}}

Scheduling \textit{classes} in Linux can have multiple \textit{policies}, and
\schedidle{} lives within the \normalclass{} class alongside the default policy
of the \normalclass{} class, which is \schednormal{}.\footnote{There is,
confusingly, also an Idle scheduling \textit{class}, but that is inaccessible to
userspace and exists solely to manage the core's transition in and out of being
actually idle (\ie{} running nothing).} The existing \schedidle{} policy is in many
ways not different from a low-weight \schednormal{} process: both are kept on
the same runqueues as all the other \schednormal{} processes, and \schedidle{}
just has a predefined low weight of 3~\cite{weight-idleprio}.

One way that \schedidle{} is promising as a foundation for our implementation of
\beclass{} is that \schedidle{} was extended to be accessible via the \cgroups{}
API recently~\cite{lkml-idle-cgroup}: a whole groups' policy can be set to
\schedidle{} via the \cgroups{} interface. This means that building on
\schedidle{} allows us to get for free the ability to use the \cgroups{} API to
mark groups as best effort. We thus have two ways to mark things as BE: we can
mark individual processes by setting their policy to \schedidle{}, or we can do
so for a whole group via the \cgroups{} API.

An additional benefit to using \schedidle{} as the starting point for our
implementation is that, after a push by Facebook, Linux developers already added
what is in effect the \entry{} check from the \beclass{}
design~\cite{fixing-idle-article}. In 2019, Linux added a check when a
\schednormal{} process becomes newly runnable on a core already running something
in \schednormal{}. This new check looks for other cores that might be currently
running a \schedidle{} process, and migrates the new process there.

\subsection{Implementating \beclass{} in \schedbe{}}

To achieve priority scheduling on top of \schedidle{}, our implementation adds
the \local{} and \exit{} parts of the \beclass{} design. We call the resulting
policy \schedbe{}. While it is technically still a policy, it implements the
desired behavior of \beclass{}, and as a result behaves as if it was a
scheduling class.

To enforce the \local{} part of the design, which calls for the local (\ie{}
single runqueue) strict prioritization of \schednormal{} over \schedbe{}
processes, we ensure that the task chosen to run from the runqueue is only
\schedbe{} if everything else on the runqueue is as well
(\autoref{ss:implementation:local}). To enforce the \exit{} check, we add a
cross-core check and potentially steal work, which, in combination with the
existing \entry{} check, ensures the global policy enforcement
(\autoref{ss:implementation:exit}).

\subsubsection{Enforcing the local policy}\label{ss:implementation:local}

In order to enforce that ruqueues only run \schedbe{} threads when there are no
runnable \schednormal{} ones, our implementation modofies Linux in two places. 

Because in existing Linux \schedidle{} and \schednormal{} share a runqueue, so
will the new \schedbe{} and \schednormal{} (doing so maintains a complicated
existing infrastructure around the runqueue for things like accounting and load
balancing). Sharing a runqueue means that the function choosing the next task
from the runqeueue will be potentially looking at a mix of both policies. This
complicates its task: we want the scheduler to enforce weights within each of
\schedbe{} and \schednormal{}, which includes keeping track of the sum of all
the weights on the runqueue, while still enforcing strict priority between them.
We add an initial check that establishes whether there are any \schednormal{}
threads on the runqueue, and skips all \schedbe{} ones if that is the case. This
strictness is what enforces parking under high load.

The second change is necessary because the first breaks an existing eligibility
mechanism. In order to help maintain fairness, Linux currently accounts for the
difference between the fair share processes should have gotten and the time they
actually got, and stores that `lag'. Processes that have gotten more time than
they should (\ie{} have negative lag), are marked as ineligible and not
considered when choosing what to run next. Since \schedbe{} threads are now
potentially not being run for a long time, there is a potential for deadlock: a
\schednormal{} task has been running for a while and accrued enough time that
its lag is negative and it is ineligible. However, if the only other thread is
\schedbe{} then we won't run that either because there is a runnable
\schednormal{} task on the runqueue. In order to avoid this situation, the
implementation removes the eligibility criterion in choosing what to run
next.


\subsubsection{Enforcing the global policy}\label{ss:implementation:exit}

Ensuring the \entry{} and \exit{} checks requires interposing on Linux's wakeup
and exit codepaths. Linux already special-cases on the wakeup path, although
only checks if the thread itself is marked as idle, and not if the group as a
whole is, and \schedbe{} checks both.

\schedbe{} also adds a check on the exit path: if the thread chosen to run next
is \schedbe{}, but the previous one was \schednormal{}, then \schedbe{} tries to
steal a queued \schednormal{} task from a different core. Specifically, it picks
the core with the max number of queued but runnable \schednormal{} threads. It
steals only one, in order to not overzealously steal. This choice mirrors what
Linux does when a core would otherwise go completely idle.
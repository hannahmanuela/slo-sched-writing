\section{Implementation}\label{s:implementation}


In order to implement \beclass{} in Linux, we build on \schedidle{}, an existing
scheduling \textit{policy} in Linux. This is because \schedidle{} already has
some of the features we want from \beclass{}.

\subsection{\schedidle{} lends itself well to \beclass{}}

Scheduling \textit{classes} in Linux can have multiple \textit{policies}, and
\schedidle{} lives within the \normalclass{} class alongside the default policy
of the \normalclass{} class, which is \schednormal{}.\footnote{There is, very
confusingly, also an Idle scheduling \textit{class}, but that not accessible to
userspace and exists solely to manage the core's transition in and out of being
actually idle (ie running nothing).} The existing \schedidle{} policy is in many
ways not different from a low weight \schednormal{} process: both are kept on
the same runqueues as all the other \schednormal{} processes, and \schedidle{}
just has a predefined low weight of 3~\cite{weight-idleprio}.

One way that \schedidle{} is promising as a foundation for our implementation of
\beclass{} is that \schedidle{} was extended to be accessible via the \cgroups{}
API recently~\cite{lkml-idle-cgroup}: a whole groups' policy can be set to
\schedidle{} via the \cgroups{} interface. This means that, if we modify
\schedidle{} to behave in the way we laid out \beclass{} should, we would get
for free the abillity to use the \cgroups{} API to mark groups as BE. We thus
have two ways to mark things as BE: we can mark individual processes by setting
their policy to \schedidle{} (ie now \schedbe{}), or we can do so whole a whole
group via the \cgroups{} API.

An additional benefit to using \schedidle{} as the foundation for our
implementation is that, after a push by Facebook, Linux developers already added
what is in effect the \entry{} check~\cite{fixing-idle-article}. In 2019, Linux
added a check when a \schednormal{} entity becomes newly runnable on a core
already running something in \schednormal{}. This new check will look for other
cores that might be currently running a \schedidle{} entity, and migrates the
new entity there.

\subsection{\schedbe{} as an implementation of \beclass{}}

To implement \beclass{}, we modify \schedidle{} to add the \local{} and \exit{}
parts of the \beclass{} design. We call the resulting policy \schedbe{}. While
it is technically still a policy, it implements the desired behavior of
\beclass{}, and as a result behaves as if it was a scheduling class. Our
implementation is a patch to Linux version 6.14.2, which implements the
\beclass{} class by extending \schedidle{} to become \schedbe{}.

To enforce the \local{} part of the design, which calls for the local (ie single
runqueue) isolation of \schedbe{} processes, we ensure that the task chosen to
run from the runqueue is only \schedbe{} if everything else on the runqueue is
as well. To enforce the \exit{} check, we add a synchronization point to
complete the global policy enforcement.

\subsection{Enforcing the local policy}

In order to enforce that ruqueues only run \schedbe{} threads when there are no
runnable \schednormal{} ones, the patch interferes in two places. 

Because in existing Linux \schedidle{} and \schednormal{} share a runqueue, so
will the new \schedbe{} and \schednormal{}. This means that the function that
chooses the next task from the runqeueue will be potentially looking at a mix of
both. We add an initial check that establishes whether there are any
\schednormal{} threads on the runqueue, and skips all \schedbe{} ones if that is
the case. 

The second change is necessary because the first throws the fair share
eligibility mechanism out of whack. In order to maintain fairness, Linux
currently accounts for the difference between the fair share processes should
have gotten and the time they actually got, and stores that `lag'. Processes
that have gotten more time than they should (ie have negative lag), are marked
as ineligible and not considered when choosing what to run next. Since
\schedbe{} threads are now potentially not being run for a long time, there is a
potential for deadlock: a \schednormal{} task has been running for a while and
accrued enough time that its lag is negative and it is ineligible. However, if
the only other thread is \schedbe{} then we won't run that either because there
is a runnable \schednormal{} task on the runqueue. In order to avoid this
situation, the patch removes the eligibility criterion in choosing what to run
next.


\subsection{Enforcing the global policy}

Ensuring the exit and entry path requires interposing on both. Linux already
special-cases on the wakeup path, although only checks if the thread itself is
marked as idle, and not if the group as a whole is, and for \schedbe{} we check
both.\hmng{this is not exactly true, but in a kinda complicated way. Do we even
want to say this?}

\schedbe{} adds a check on the exit: if the thread chosen to run next is
\schedbe{}, then the patch tries to steal a queued \schednormal{} task from a
different core. Specifically, it picks the core with the max number of queued
but runnable \schednormal{} threads --- but only steals one, in order to not
overzealously steal. Linux does the same thing when a core would otherwise go
completely idle.
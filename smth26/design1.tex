\section{Design of \sys}

\sys is a new design for scheduling processes that enforces group
weights across cores.  It assign each runnable process a virtual based
on its weight and maintains a global heap of virtual runtimes, with
the process with lowest virtual at the front of the heap. To implement
this global heap in a scalable manner, \sys uses an {\it approximate}
global heap and may select process that has a virtual time slightly
higher than the absolute minimum virtual time.  Finally, if several
processes of the same group are runnable, \sys selects the one that it
has run recently to provide core affinity so that the selected process
runs with a warm L1/L2 caches.

\subsection{Virtual time}

\begin{figure}
\begin{Verbatim}[commandchars=\\\{\},numbers=left,firstnumber=1,stepnumber=1,codes={\catcode`\$=3\catcode`\^=7\catcode`\_=8\relax},fontsize=\scriptsize,numbersep=6pt,xleftmargin=0.2in]
\PY{k}{struct}\PY{+w}{ }\PY{n+nc}{group}\PY{+w}{ }\PY{p}{\PYZob{}}
\PY{+w}{  }\PY{n}{w\PYZus{}t}\PY{+w}{ }\PY{n}{weight}\PY{p}{;}
\PY{+w}{  }\PY{n}{vt\PYZus{}t}\PY{+w}{ }\PY{n}{vruntime}\PY{p}{;}
\PY{+w}{  }\PY{n}{vt\PYZus{}t}\PY{+w}{ }\PY{n}{min\PYZus{}vt\PYZus{}deq}\PY{p}{;}
\PY{p}{\PYZcb{}}
\end{Verbatim}

\caption{The state for each group}
\label{fig:group}
\end{figure}

\autoref{fig:group} shows the state that \sys maintains for each
cgroup: \cc{weight}, the groups weight, \cc{vruntime}, which is the
next available virtual runtime for a process in the cgroup to run, and
\cc{min_vt_deq}, which is the virtual runtime when a group goes to
  ``sleep'' (i.e., when all group's processes are sleeping).

\begin{figure}
\begin{Verbatim}[commandchars=\\\{\},numbers=left,firstnumber=1,stepnumber=1,codes={\catcode`\$=3\catcode`\^=7\catcode`\_=8\relax},fontsize=\scriptsize,numbersep=6pt,xleftmargin=0.2in]
\PY{n}{vt\PYZus{}t}\PY{+w}{ }\PY{n+nf}{calc\PYZus{}delta}\PY{p}{(}\PY{n}{vt\PYZus{}t}\PY{+w}{ }\PY{n}{delta}\PY{p}{,}\PY{+w}{ }\PY{n}{w\PYZus{}t}\PY{+w}{ }\PY{n}{weight}\PY{p}{)}\PY{+w}{ }\PY{p}{\PYZob{}}
\PY{+w}{        }\PY{k}{return}\PY{+w}{ }\PY{n}{delta}\PY{+w}{ }\PY{o}{/}\PY{+w}{ }\PY{n}{weight}\PY{p}{;}
\PY{p}{\PYZcb{}}

\PY{k+kt}{void}\PY{+w}{ }\PY{n+nf}{enq\PYZus{}proc\PYZus{}vt}\PY{p}{(}\PY{k}{struct}\PY{+w}{ }\PY{n+nc}{global\PYZus{}heap}\PY{+w}{ }\PY{o}{*}\PY{n}{gh}\PY{p}{,}\PY{+w}{ }\PY{k}{struct}\PY{+w}{ }\PY{n+nc}{process}\PY{+w}{ }\PY{o}{*}\PY{n}{p}\PY{p}{,}\PY{+w}{ }\PY{k}{struct}\PY{+w}{ }\PY{n+nc}{heap}\PY{+w}{ }\PY{o}{*}\PY{n}{h}\PY{p}{)}\PY{+w}{ }\PY{p}{\PYZob{}}
\PY{+w}{  }\PY{n}{vt\PYZus{}t}\PY{+w}{ }\PY{n}{wvt}\PY{+w}{ }\PY{o}{=}\PY{+w}{ }\PY{n}{calc\PYZus{}delta}\PY{p}{(}\PY{n}{gh}\PY{o}{\PYZhy{}}\PY{o}{\PYZgt{}}\PY{n}{tick\PYZus{}length}\PY{p}{,}\PY{+w}{ }\PY{n}{p}\PY{o}{\PYZhy{}}\PY{o}{\PYZgt{}}\PY{n}{he}\PY{p}{.}\PY{n}{weight}\PY{p}{)}\PY{p}{;}
\PY{+w}{  }\PY{n}{vt\PYZus{}t}\PY{+w}{ }\PY{n}{my\PYZus{}vt}\PY{+w}{ }\PY{o}{=}\PY{+w}{ }\PY{n}{atomic\PYZus{}fetch\PYZus{}add}\PY{p}{(}\PY{o}{\PYZam{}}\PY{n}{p}\PY{o}{\PYZhy{}}\PY{o}{\PYZgt{}}\PY{n}{group}\PY{o}{\PYZhy{}}\PY{o}{\PYZgt{}}\PY{n}{vruntime}\PY{p}{,}\PY{+w}{ }\PY{n}{vt}\PY{p}{)}\PY{p}{;}
\PY{+w}{  }\PY{n}{p}\PY{o}{\PYZhy{}}\PY{o}{\PYZgt{}}\PY{n}{he}\PY{p}{.}\PY{n}{vruntime}\PY{+w}{ }\PY{o}{=}\PY{+w}{ }\PY{n}{my\PYZus{}vt}\PY{p}{;}
\PY{+w}{  }\PY{n}{mh\PYZus{}add\PYZus{}process}\PY{p}{(}\PY{n}{p}\PY{p}{,}\PY{+w}{ }\PY{n}{h}\PY{p}{)}\PY{p}{;}
\PY{p}{\PYZcb{}}
\end{Verbatim}

\caption{Enqueue a process}
\label{fig:enq}
\end{figure}

When a process of in a group becomes runnable, \sys assigns it a
virtual and inserts it in the global heap, as show in
\autoref{fig:enq}. Consider a \cc{tick_length} of 1000us and two
groups, $g_1$ with weight 10 and $g_2$ with weight 20. When a process
of $g_1$ becomes runnable it will get assigned the current group's
virtual time and increases it by 100, $ = 1000/10$.  If concurrently,
on another core another process of $g_1$ becomes runnable it will be
assigned a virtual time that is 100 larger than the first process, and
$g_1$'s virtual time will be incremented by 100, ensuring the
processes are correctly spaced by 100 in virtual time.

When a process of $g_2$ becomes runnable, it will increases its
group's virtual time by 50, $ = 1000/20$, and the processes of $g_2$
will be spaced by 50 in virtual time.  If we look over a window of a
1000 in virtual time, $g_1$ will run 10 times and $g_2$ will run 20
times, reflecting their relative weight of 10 and 20.

\begin{figure}
\begin{Verbatim}[commandchars=\\\{\},numbers=left,firstnumber=1,stepnumber=1,codes={\catcode`\$=3\catcode`\^=7\catcode`\_=8\relax},fontsize=\scriptsize,numbersep=6pt,xleftmargin=0.2in]
\PY{k+kt}{void}\PY{+w}{ }\PY{n+nf}{grp\PYZus{}adjust\PYZus{}vruntime}\PY{p}{(}\PY{k}{struct}\PY{+w}{ }\PY{n+nc}{global\PYZus{}heap}\PY{+w}{ }\PY{o}{*}\PY{n}{gh}\PY{p}{,}\PY{+w}{ }\PY{k}{struct}\PY{+w}{ }\PY{n+nc}{process}\PY{+w}{ }\PY{o}{*}\PY{n}{p}\PY{p}{,}\PY{+w}{ }\PY{n}{t\PYZus{}t}\PY{+w}{ }\PY{n}{time\PYZus{}passed}\PY{p}{)}\PY{+w}{ }\PY{p}{\PYZob{}}
\PY{+w}{  }\PY{n}{vt\PYZus{}t}\PY{+w}{ }\PY{n}{vt}\PY{+w}{ }\PY{o}{=}\PY{+w}{ }\PY{n}{calc\PYZus{}delta}\PY{p}{(}\PY{n}{time\PYZus{}passed}\PY{p}{,}\PY{+w}{ }\PY{n}{p}\PY{o}{\PYZhy{}}\PY{o}{\PYZgt{}}\PY{n}{he}\PY{p}{.}\PY{n}{weight}\PY{p}{)}\PY{p}{;}
\PY{+w}{  }\PY{n}{vt\PYZus{}t}\PY{+w}{ }\PY{n}{wvt}\PY{+w}{ }\PY{o}{=}\PY{+w}{ }\PY{n}{calc\PYZus{}delta}\PY{p}{(}\PY{n}{gh}\PY{o}{\PYZhy{}}\PY{o}{\PYZgt{}}\PY{n}{tick\PYZus{}length}\PY{p}{,}\PY{+w}{ }\PY{n}{p}\PY{o}{\PYZhy{}}\PY{o}{\PYZgt{}}\PY{n}{he}\PY{p}{.}\PY{n}{weight}\PY{p}{)}\PY{p}{;}
\PY{+w}{  }\PY{k}{if}\PY{+w}{ }\PY{p}{(}\PY{n}{wvt}\PY{+w}{ }\PY{o}{\PYZgt{}}\PY{+w}{ }\PY{n}{vt}\PY{p}{)}\PY{+w}{ }\PY{p}{\PYZob{}}
\PY{+w}{    }\PY{n}{atomic\PYZus{}fetch\PYZus{}add}\PY{p}{(}\PY{o}{\PYZam{}}\PY{n}{p}\PY{o}{\PYZhy{}}\PY{o}{\PYZgt{}}\PY{n}{group}\PY{o}{\PYZhy{}}\PY{o}{\PYZgt{}}\PY{n}{vruntime}\PY{p}{,}\PY{+w}{ }\PY{o}{\PYZhy{}}\PY{p}{(}\PY{n}{wvt}\PY{o}{\PYZhy{}}\PY{n}{vt}\PY{p}{)}\PY{p}{)}\PY{p}{;}
\PY{+w}{  }\PY{p}{\PYZcb{}}
\PY{p}{\PYZcb{}}
\end{Verbatim}

\caption{Adjust virtual runtime when a process yield before a full clock tick.}
\label{fig:adjust}
\end{figure}

If a process yields its core before \cc{tick_length}, it adjusts the
group's virtual time to reflect it didn't use the full tick, as shown
in \autoref{fig:adjust}.  The effect of adjusting is to move the
group's virtual time backwards so that the next process of the group
inherits the time that the first process didn't use.

The way \sys computes virtual time is similar to Linux's fair-share
scheduler with the main difference being that \sys must account for
multiple cores concurrently assigning virtual times to processes of
the same group.  Since in Linux virtual time is per run queue, and
thus updated only by one core, it doesn't need to handle concurrent
updates. It, for example, doesn't update the group's virtual
on enqueue and adjust it if a process doesn't run for its full tick.

\begin{figure}
\begin{Verbatim}[commandchars=\\\{\},numbers=left,firstnumber=1,stepnumber=1,codes={\catcode`\$=3\catcode`\^=7\catcode`\_=8\relax},fontsize=\scriptsize,numbersep=6pt,xleftmargin=0.2in]
\PY{k}{struct}\PY{+w}{ }\PY{n+nc}{process}\PY{+w}{ }\PY{o}{*}\PY{n}{min\PYZus{}affinity}\PY{p}{(}\PY{k}{struct}\PY{+w}{ }\PY{n+nc}{core}\PY{+w}{ }\PY{o}{*}\PY{n}{c}\PY{p}{)}\PY{+w}{ }\PY{p}{\PYZob{}}
\PY{+w}{  }\PY{k}{struct}\PY{+w}{ }\PY{n+nc}{process}\PY{+w}{ }\PY{o}{*}\PY{n}{cp}\PY{+w}{ }\PY{o}{=}\PY{+w}{ }\PY{n}{c}\PY{o}{\PYZhy{}}\PY{o}{\PYZgt{}}\PY{n}{process}\PY{p}{;}
\PY{+w}{  }\PY{k}{struct}\PY{+w}{ }\PY{n+nc}{heap}\PY{+w}{ }\PY{o}{*}\PY{n}{h}\PY{+w}{ }\PY{o}{=}\PY{+w}{ }\PY{n}{cp}\PY{o}{\PYZhy{}}\PY{o}{\PYZgt{}}\PY{n}{h}\PY{p}{;}
\PY{+w}{  }\PY{k+kt}{int}\PY{+w}{ }\PY{n}{cid}\PY{+w}{ }\PY{o}{=}\PY{+w}{ }\PY{n}{atomic\PYZus{}load}\PY{p}{(}\PY{o}{\PYZam{}}\PY{n}{cp}\PY{o}{\PYZhy{}}\PY{o}{\PYZgt{}}\PY{n}{cid}\PY{p}{)}\PY{p}{;}
\PY{+w}{  }\PY{k}{if}\PY{+w}{ }\PY{p}{(}\PY{n}{cid}\PY{+w}{ }\PY{o}{!}\PY{o}{=}\PY{+w}{ }\PY{n}{c}\PY{o}{\PYZhy{}}\PY{o}{\PYZgt{}}\PY{n}{cid}\PY{p}{)}\PY{+w}{ }\PY{p}{\PYZob{}}
\PY{+w}{    }\PY{k}{return}\PY{+w}{ }\PY{n+nb}{NULL}\PY{p}{;}
\PY{+w}{  }\PY{p}{\PYZcb{}}
\PY{+w}{  }\PY{k}{if}\PY{p}{(}\PY{n}{atomic\PYZus{}load}\PY{p}{(}\PY{o}{\PYZam{}}\PY{n}{h}\PY{o}{\PYZhy{}}\PY{o}{\PYZgt{}}\PY{n}{heap}\PY{p}{[}\PY{l+m+mi}{0}\PY{p}{]}\PY{p}{.}\PY{n}{elem}\PY{p}{)}\PY{+w}{ }\PY{o}{!}\PY{o}{=}\PY{+w}{ }\PY{n}{cp}\PY{p}{)}\PY{+w}{ }\PY{p}{\PYZob{}}
\PY{+w}{    }\PY{k}{return}\PY{+w}{ }\PY{n+nb}{NULL}\PY{p}{;}
\PY{+w}{  }\PY{p}{\PYZcb{}}
\PY{+w}{  }\PY{k}{struct}\PY{+w}{ }\PY{n+nc}{process}\PY{+w}{ }\PY{o}{*}\PY{n}{p}\PY{+w}{ }\PY{o}{=}\PY{+w}{ }\PY{n+nb}{NULL}\PY{p}{;}
\PY{+w}{  }\PY{n}{lock\PYZus{}acquire}\PY{p}{(}\PY{o}{\PYZam{}}\PY{n}{h}\PY{o}{\PYZhy{}}\PY{o}{\PYZgt{}}\PY{n}{lk}\PY{p}{)}\PY{p}{;}
\PY{+w}{  }\PY{k}{if}\PY{p}{(}\PY{p}{(}\PY{n}{h}\PY{o}{\PYZhy{}}\PY{o}{\PYZgt{}}\PY{n}{heap}\PY{p}{[}\PY{l+m+mi}{0}\PY{p}{]}\PY{p}{.}\PY{n}{elem}\PY{+w}{ }\PY{o}{!}\PY{o}{=}\PY{+w}{ }\PY{n}{cp}\PY{p}{)}\PY{+w}{ }\PY{o}{|}\PY{o}{|}\PY{+w}{ }\PY{p}{(}\PY{n}{cp}\PY{o}{\PYZhy{}}\PY{o}{\PYZgt{}}\PY{n}{cid}\PY{+w}{ }\PY{o}{!}\PY{o}{=}\PY{+w}{ }\PY{n}{c}\PY{o}{\PYZhy{}}\PY{o}{\PYZgt{}}\PY{n}{cid}\PY{p}{)}\PY{p}{)}\PY{+w}{ }\PY{p}{\PYZob{}}
\PY{+w}{    }\PY{k}{goto}\PY{+w}{ }\PY{n}{end}\PY{p}{;}
\PY{+w}{  }\PY{p}{\PYZcb{}}
\PY{n+nl}{retry}\PY{p}{:}
\PY{+w}{  }\PY{n}{vt\PYZus{}t}\PY{+w}{ }\PY{n}{vt}\PY{p}{;}
\PY{+w}{  }\PY{k+kt}{int}\PY{+w}{ }\PY{n}{j}\PY{+w}{ }\PY{o}{=}\PY{+w}{ }\PY{n}{mh\PYZus{}rand\PYZus{}heap}\PY{p}{(}\PY{n}{cp}\PY{o}{\PYZhy{}}\PY{o}{\PYZgt{}}\PY{n}{mh}\PY{p}{,}\PY{+w}{ }\PY{n}{c}\PY{p}{,}\PY{+w}{ }\PY{n}{h}\PY{o}{\PYZhy{}}\PY{o}{\PYZgt{}}\PY{n}{id}\PY{p}{)}\PY{p}{;}
\PY{+w}{  }\PY{k}{struct}\PY{+w}{ }\PY{n+nc}{heap}\PY{+w}{ }\PY{o}{*}\PY{n}{h1}\PY{+w}{ }\PY{o}{=}\PY{+w}{ }\PY{n}{select\PYZus{}affinity}\PY{p}{(}\PY{n}{h}\PY{o}{\PYZhy{}}\PY{o}{\PYZgt{}}\PY{n}{id}\PY{p}{,}\PY{+w}{ }\PY{n}{j}\PY{p}{,}\PY{+w}{ }\PY{o}{\PYZam{}}\PY{n}{vt}\PY{p}{)}\PY{p}{;}
\PY{+w}{  }\PY{k}{if}\PY{+w}{ }\PY{p}{(}\PY{n}{h1}\PY{+w}{ }\PY{o}{=}\PY{o}{=}\PY{+w}{ }\PY{n}{h}\PY{p}{)}\PY{+w}{ }\PY{p}{\PYZob{}}
\PY{+w}{    }\PY{n}{p}\PY{+w}{ }\PY{o}{=}\PY{+w}{ }\PY{n}{remove\PYZus{}min}\PY{p}{(}\PY{n}{h}\PY{p}{)}\PY{p}{;}
\PY{+w}{    }\PY{k}{goto}\PY{+w}{ }\PY{n}{end}\PY{p}{;}
\PY{+w}{  }\PY{p}{\PYZcb{}}
\PY{+w}{  }\PY{k+kt}{int}\PY{+w}{ }\PY{n}{l}\PY{+w}{ }\PY{o}{=}\PY{+w}{ }\PY{n}{lock\PYZus{}try\PYZus{}acquire}\PY{p}{(}\PY{o}{\PYZam{}}\PY{n}{h1}\PY{o}{\PYZhy{}}\PY{o}{\PYZgt{}}\PY{n}{lk}\PY{p}{)}\PY{p}{;}
\PY{+w}{  }\PY{k}{if}\PY{+w}{ }\PY{p}{(}\PY{n}{l}\PY{+w}{ }\PY{o}{!}\PY{o}{=}\PY{+w}{ }\PY{l+m+mi}{0}\PY{p}{)}\PY{+w}{ }\PY{p}{\PYZob{}}
\PY{+w}{    }\PY{k}{goto}\PY{+w}{ }\PY{n}{retry}\PY{p}{;}
\PY{+w}{  }\PY{p}{\PYZcb{}}
\PY{+w}{  }\PY{n}{vt\PYZus{}t}\PY{+w}{ }\PY{n}{vt0}\PY{+w}{ }\PY{o}{=}\PY{+w}{ }\PY{n}{h1}\PY{o}{\PYZhy{}}\PY{o}{\PYZgt{}}\PY{n}{heap}\PY{p}{[}\PY{l+m+mi}{0}\PY{p}{]}\PY{p}{.}\PY{n}{vruntime}\PY{p}{;}
\PY{+w}{  }\PY{k}{if}\PY{+w}{ }\PY{p}{(}\PY{n}{vt}\PY{+w}{ }\PY{o}{!}\PY{o}{=}\PY{+w}{ }\PY{n}{vt0}\PY{p}{)}\PY{+w}{ }\PY{p}{\PYZob{}}
\PY{+w}{    }\PY{n}{lock\PYZus{}release}\PY{p}{(}\PY{o}{\PYZam{}}\PY{n}{h1}\PY{o}{\PYZhy{}}\PY{o}{\PYZgt{}}\PY{n}{lk}\PY{p}{)}\PY{p}{;}
\PY{+w}{    }\PY{k}{goto}\PY{+w}{ }\PY{n}{retry}\PY{p}{;}
\PY{+w}{  }\PY{p}{\PYZcb{}}\PY{+w}{ }
\PY{+w}{  }\PY{n}{p}\PY{+w}{ }\PY{o}{=}\PY{+w}{ }\PY{n}{remove\PYZus{}min}\PY{p}{(}\PY{n}{h1}\PY{p}{)}\PY{p}{;}
\PY{+w}{  }\PY{n}{lock\PYZus{}release}\PY{p}{(}\PY{o}{\PYZam{}}\PY{n}{h1}\PY{o}{\PYZhy{}}\PY{o}{\PYZgt{}}\PY{n}{lk}\PY{p}{)}\PY{p}{;}
\PY{n+nl}{end}\PY{p}{:}
\PY{+w}{  }\PY{n}{lock\PYZus{}release}\PY{p}{(}\PY{o}{\PYZam{}}\PY{n}{h}\PY{o}{\PYZhy{}}\PY{o}{\PYZgt{}}\PY{n}{lk}\PY{p}{)}\PY{p}{;}
\PY{+w}{  }\PY{k}{return}\PY{+w}{ }\PY{n}{p}\PY{p}{;}
\PY{p}{\PYZcb{}}
\end{Verbatim}

\caption{The schedule function}
\label{fig:schedule}
\end{figure}



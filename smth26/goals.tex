%-------------------------------------------------------------------------------
\section{Goals}
\label{s:goals}
%-------------------------------------------------------------------------------


Our goal is to allow for the different quality of service (QOS) classes that
Kubernetes uses, and to enforce guarantees for each of the QOS classes.

% maybe I should do like a survey of existing systems and look at how they deal
% with it?

The highest QOS class is Guaranteed. Guaranteed services are guaranteed access
to the number of CPUs they request, but are also limited to them. Guaranteed
services should not be able to tell the difference if unused but reserved cores
are being used to run other workloads or not. They should never be able to use
more CPUs than they reserved.

The middle QOS class is Burstable. Burstable services are also guaranteed access
to the number of CPUs they requested, with the same stipulation that they not be
able to tell the difference if unused but reserved cores are being used to run
other work. Burstable services are however allowed to use CPUs beyond their
reservation opportunistically.\hmng{where does the limit on bursting come in? I
think in amazon the whole thing is credit-based and there is no minimum,
although its also not clear at what level that is/needs to be enforced}

The lowest QOS class is BestEffort. BestEffort services are only ever allowed to
run opportunistically, and when they do they have a priority below that of
bursting Burstable services.

All of these classes are defined by guaranteed vCPU minimums, and limits. In
Kubernetes these are expressed by attaching resource `requests' and `limits' to
pods. These are expressed as a number of vCPUs, and can be fractional (\ie{} a
pod could request 0.5 vCPUs).


%-------------------------------------------------------------------------------
\section{Goals}
\label{s:goals}
%-------------------------------------------------------------------------------


Developers use the different quality of service (QOS) classes that Kubernetes
supports to run different services with different latency needs.

The highest QOS class is Guaranteed. Guaranteed services are guaranteed access
to the number of CPUs they request, but are also limited to them. They are the
last to be evicted under resource pressure, and are used for critical
infrastructure like databases or core microservices~\cite{reddit-kub-qos}.

The middle QOS class is Burstable. Burstable services are similarly guaranteed
access to the number of CPUs they requested, but are also allowed to use CPUs
beyond their reservation opportunistically. This is the category in which a web
server would be expected to run, where it can make use of the bursting in
moments of higher load.

Because both Guaranteed and Burstable pods have reserved CPUs, developers choose
reservations such that they are enough to handle the expected load. That way,
even when overall load is high in the cluster, the important services have
access to the resources they requested and their latencies stay stable.

The lowest QOS class is BestEffort. BestEffort services are only allowed to run
on resources that would otherwise go idle, and when they do they have a priority
below that of bursting Burstable services. This is the QOS class that a
developer would run background data analytics or image resize jobs, where the
time to completion is inconsequential, within bounds.

In Kubernetes, the reservations and limits are expressed by attaching resource
`requests' and `limits' to pods. These are expressed as a number of vCPUs, and
can be fractional (\ie{} a pod could request 0.5 vCPUs). Pods are BestEffort if
their request is 0, and are Guaranteed if they have both a request and a limit
and those two are the same.


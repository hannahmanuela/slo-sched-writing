\documentclass[letterpaper,twocolumn,10pt]{article}
\PassOptionsToPackage{hyphens}{url}
\usepackage[style=base]{caption}
\usepackage[10pt,nocopyright]{sigmin}
\PassOptionsToPackage{pdfa, breaklinks, pdfborder={0 0 0}}{hyperref}

\usepackage[square,comma,numbers,sort&compress]{natbib}

% to be able to draw some self-contained figs
\usepackage{times}
\usepackage{tikz}
\usepackage{amsmath}
\usepackage{amssymb}
\usepackage{hyperref}
\usepackage[normalem]{ulem}
\usepackage{listings}
\usepackage{xspace}
\usepackage{booktabs}
\usepackage{multirow}
\usepackage{caption}
\usepackage{subcaption}
\usepackage{enumitem}

\usepackage{xcolor}
\usepackage{verbatim}
\usepackage{fancyvrb}
\usepackage{listings}
\usepackage[T1]{fontenc}

\usepackage{relsize}


\usepackage[utf8]{inputenc}
% \usepackage[compact, small]{titlesec}

\newcommand{\sys}{\textsc{GlobalHeap}\xspace}
\newcommand{\mheap}{\textsc{mheap}\xspace}

\newcommand{\schedidle}{\texttt{sched\_idle}}
\newcommand{\schednormal}{\texttt{sched\_normal}}
\newcommand{\schedfifo}{\texttt{sched\_fifo}}
\newcommand{\schedrr}{\texttt{sched\_rr}}
\newcommand{\cgroups}{\texttt{cgroups}}
\newcommand{\schedbe}{\texttt{sched\_be}}

\newcommand{\beclass}{\texttt{BeClass}}
\newcommand{\normalclass}{\texttt{Normal}}
\newcommand{\rtclass}{\texttt{RTClass}}
\newcommand{\deadlineclass}{\texttt{Deadline}}

\newcommand{\local}{\textit{local}}
\newcommand{\entry}{\textit{entry}}
\newcommand{\exit}{\textit{exit}}


\newcommand{\eg}{{e.g.},\xspace}
\newcommand{\ie}{{i.e.},\xspace}

\newcommand\hmng[1]{\textcolor{blue!40!red}{[hmng: {#1}]}}

\def\Snospace~{\S{}}
\def\sectionautorefname{\Snospace}
\def\subsectionautorefname{\Snospace}
\def\subsubsectionautorefname{\Snospace}

\definecolor{codegreen}{rgb}{0,0.4,0}
\definecolor{codegray}{rgb}{0.5,0.5,0.5}
\definecolor{codepurple}{rgb}{0.58,0,0.82}
\definecolor{backcolour}{rgb}{0.95,0.95,0.92}

\newcommand{\cc}[1]{\texorpdfstring{\mbox{\smaller[0.5]\texttt{\detokenize{#1}}}}{\texttt{\detokenize{#1}}}}

\input{code/fmt.tex}

%-------------------------------------------------------------------------------
\begin{document}
%-------------------------------------------------------------------------------

%don't want date printed
\date{}

%%
%% The "title" command has an optional parameter,
%% allowing the author to define a "short title" to be used in page headers.
% make title bold and 14 pt font (Latex default is non-bold, 16 pt)
\title{XX}

%%
%% The "author" command and its associated commands are used to define
%% the authors and their affiliations.
%% Of note is the shared affiliation of the first two authors, and the
%% "authornote" and "authornotemark" commands
%% used to denote shared contribution to the research.

\author{
{\rm Anonymous Authors}\\
} % end author

%\author{Hannah Gross}
%\affiliation{%
  %\institution{MIT}
%\email{hannah@csail.mit.edu}
%\affiliation{%
%  \institution{MIT}
%  \state{}
  %\country{}
%}

\maketitle

%-------------------------------------------------------------------------------
\section{Introduction}
%-------------------------------------------------------------------------------

Developers use Service Level Objectives (SLOs) between teams and Service Level
Agreements (SLAs) towards customers, which represent guarantees about uptime and
maximal latencies of the API that team/company is in charge of~\cite{awssla}.
Monitoring systems watch performance and page on-call developers if these
guarantees are not being met~\cite{cloudwatch}.

However, the interface to many scheduling frameworks, such as
Kubernetes~\cite{kubernetes} and many research systems~\cite{caladan}, offers
developers two options: latency-critical (LC) applications have concrete
reservations, and jobs with no immediate deadline can be run as best-effort (BE)
tasks. The scheduler bin-packs LC work, and runs BE work opportunistically.


For a developer to translate their requirements into this interface requires two
steps: pick a category (LC or BE), and, if the its the former, generate a
concrete reservation. Both of these steps are difficult: some work might not be
completely LC or BE, and reservations require the developer to make estimations
about peak load, and in fact incentivizes them to
overestimate~\cite*{overprovision} --- making achieving high utilization hard.

Imagine a web developer, whose website has four different types of work it has
to do: \\
(1) load static pages (eg the homepage) --- shortest and very time critical; 
(2) load dynamic pages (eg a users profile page) --- slightly longer and less
time critical;
(3) foreground data processing (eg processing a user uploaded file of image) ---
requires a fair amount of processing but still user-facing and thus latency
sensitive;
(4) background data processing (eg updating a data warehouse) --- runs overnight
and just needs to finish by morning.

The only candidate for BE is (4). The other three are user-facing and as such
are LC and require reservations. But among 1-3 there are levels of
criticality: it is essential that the homepage load time be low and constant,
but processing a user upload can take more time during high load.

This submission presents \textit{\sysname}, a SLA-based scheduling system.
Central to \sysname{} is the observation that the priorities of utilization and
latency don't have to be opposing.\ \sysname{} changes the interface with which
resource requirements are communicated: it makes \textit{deadlines} and
\textit{maximum compute times} the central metrics provided, thereby
side-stepping the LC/BE binary, and making the developer's requirements explicit
to the scheduler.

In \sysname{}, developers submit jobs, attached with the maximum execution time
as well as a deadline. In the website example, rather than estimate load for
each job, the resources required to run it, then add 20\% padding, developers
can simply give the scheduler the handler for each endpoint, and attach to each
its deadline and an experiential maximum compute time.
%-------------------------------------------------------------------------------
\section{Goals}
\label{s:goals}
%-------------------------------------------------------------------------------


Developers use the different quality of service (QOS) classes that Kubernetes
supports to run different services with different latency needs. We expect a
multi-tenant setting, where developers have to pay for the resources they use.

The highest QOS class is Guaranteed. Guaranteed services are guaranteed access
to the number of CPUs they request, but are also limited to them. Developers
using this class ned to provision and thus pay for peak load, and in return get
guaranteed access to enough resources to handle peak load. Examples include
critical infrastructure like databases~\cite{reddit-kub-qos}.

The middle QOS class is Burstable. Burstable services are similarly guaranteed
access to the number of CPUs they requested, but are also allowed to use CPUs
beyond their reservation opportunistically. This is the category in which a web
server would be expected to run, where it can make use of the bursting in
moments of higher load. As a result, developers can provision for closer to
average load. If the number of services bursting at the same time is low, each
will be able to use slack of other services to handle the bursting load.
However, it is not guaranteed that the resources will be available.

The lowest QOS class is BestEffort. BestEffort services are only allowed to run
on resources that would otherwise go idle. For instance, if a CPU has completely
run out of Burstable and Guaranteed things to run, it should run BestEffort
threads. This is the QOS class that a developer would run background data
analytics or image resize jobs, where the time to completion is inconsequential.

In Kubernetes, the reservations and limits are expressed by attaching resource
`requests' and `limits' to pods. These are expressed as a number of vCPUs, and
can be fractional (\ie{} a pod could request 0.5 vCPUs). Pods are BestEffort if
their request is 0, and are Guaranteed if they have both a request and a limit
and those two are the same.


%-------------------------------------------------------------------------------
\section{Problem}\label{s:problem}
%-------------------------------------------------------------------------------

This section shows that the observed behavior when running
applications on Kubernetes violates the expectation of stable
performance for high QOS class pods: Guaranteed as well as Burstable
pods see latency increases based on load on lower classes. We
track the issue down to Linux' \cgroups{}, and show that its
implementation is not conducive to supporting the classes, and
violates the behavior that the \cgroups{} documentation specifies.

We use as a case study a small but realistic social network web application,
which we run using Kubernetes. We use a CPU-bound server for Guaranteed or
Burstable, and an image resize job for the BestEffort QOS class. In all cases,
we run two pods, each requesting 2 cores, on the same four cores.

\subsection{BestEffort pods impact Burstable and Guaranteed performance}

A goal of the QOS classes is that the lowest class, BestEffort, be
invisible to higher classes. BestEffort tasks need to be preempted as soon as a
higher-class pod has anything to run. If this is correctly enforced, Guaranteed
and Burstable pods should be able to maintain steady average and tail latency
in the face of changing load on a BestEffort job.

\begin{figure}[t]
    \centering
    \includegraphics[width=\columnwidth]{graphs/kubernetes-unedited.png}
    \caption{Running a BestEffort affects the server's
    performance}\label{fig:kubernetes-unedited}
\end{figure}

However, we find that running a BestEffort job does in fact impact the
performance of higher class pods. \autoref{fig:kubernetes-unedited} shows the
result of running a best effort workload, using the BestEffort class, alongside
the web server, running as a Burstable pod. The top graph plots the end-to-end
latency of an endpoint that gets a users feed and applies a moderation; the
bottom graph shows the throughput of a best effort image resize job. After the
image resize job starts running, the mean response latency of the web
application jumps from $\sim$7ms to $\sim$15ms, and the 99th percentile latency
from $\sim$10ms to $\sim$25ms. The graph looks the same if the web server is
running as a Guaranteed pod. This clearly violates the goal that Guaranteed and
Burstable pods get access to reserved CPUs when they could use them.


\begin{figure}[t]
    \centering
    \includegraphics[width=\columnwidth]{graphs/schedviz-be-problem.png}
    \caption{Core 6 runs an image resize process, unaware that cores 2, 4, and 8
    all have runnable and queued server threads}\label{fig:schedviz-be-problem}
\end{figure}

To understand the cause of the above violation,
we visualize the
trace of the experiment using schedviz~\cite{schedviz-tool}. What we find is
that frequently \textit{one core is running a best effort process, while threads
of processes with resource reservations are queued on another}.
\autoref{fig:schedviz-be-problem} shows an outtake from the trace. The process
running on each core is shown as an oval, and queued processes are shown as
rectangles below; the x-axis is time and the y-axis shows the 4 cores the
experiment is running on. We see that on one of the cores, the red process that
is running for the whole 10ms is a thread of the image resize job, while server
threads, shown in varying shades of blue, are queued on the other cores.

Understanding why this is happening requires understanding how Kubernetes
enforces the CPU isolation between pods. Kubernetes uses the \cgroups{} weight
interface to enforce pods' CPU requests: It makes sure that machines are never
oversubscribed on requested CPU, and then uses weights to split up the cores on
the machine.

Burstable and Guaranteed pods are given a weight that is calculated based on the
number of cores the machine has and how many CPUs the pod requested. For
instance, on a 56 core machine we found that a pod that requested 4 cores was
given a weight of 157, whereas one that request 2 CPUs was given a weight of 79.
Kubernetes uses the smallest weight possible, 1, to run BestEffort pods.

The key fact relevant is that Linux implements \cgroups{} weight as a
{\it local} property: Linux maintains a separate run-queue on each core, and within
each run-queue, Linux's scheduler works to maintain the correct ratio of received
CPU time at each scheduling. Linux, however, does not enforce the weight ratios across
cores.

The observed inversion happens then because the threading model of the server
interacts with the per-core run-queues. The server uses a pool of worker threads
(one per client). The number of threads that the server has is larger than the
number of cores, which means that each core has multiple server threads. It
occasionally occurs that all the current requests are on server threads that
happen to be on only three of the four available cores, which leaves one
runqeueue with only idle server worker threads. This leads to an inversion,
where the core that has no runnable high weight server threads and thus runs a
low weight image resize thread, even while other cores have queued high weight
processes.


\subsection{Burstable pods affect Guaranteed pods}

Another relevant promise of the QOS-class-based interface is that two pods with
reservations not affect each other. Because Kubernetes never oversubscribes on
requests, that means that every pod with a reservation should be able to get
access to the CPUs it requested, and use them. We would then expect that a
Guaranteed pod is able to maintain steady average and tail latency under steady
load, irrespective of the load on a Burstable pod running alongside it.


\begin{figure}[t]
    \centering
    \includegraphics[width=\columnwidth]{graphs/kubernetes-lc-burst.png}
    \caption{In Kubernetes, running a Burstable service2 alongside the Guaranteed
    service1 affects service1's performance}\label{fig:kubernetes-lc-burst}
\end{figure}

To test this expectation, we run experiments using two servers: service1 runs
the same web server, and service2 is a simple CPU-bound server. 
We run the web server in the Guaranteed class, and service2 in the Burstable
class. Surprisingly, we see a significant latency impact under high load.
\autoref{fig:kubernetes-lc-burst} shows the web server's latency and the overall
utilization, before and after starting load on service2. We see that average
latency jumps from $\sim$6ms to $\sim$9.5, and the 99th pctile from 7ms to 22.

\begin{figure}[t]
    \centering
    \includegraphics[width=\columnwidth]{graphs/schedviz-lc-burst-problem.png}
    \caption{When service1 is running alone, it has all the cores to itself, but
    after service2 starts it gets only half of whatever core it runs on, even if
    it isn't using all the cores available and service2 is running uncontended
    on the other cores}\label{fig:schedviz-lc-burst-problem}
\end{figure}

\autoref{fig:schedviz-lc-burst-problem} shows an outtake from the trace of the
experiment. Threads of service1 are colored in, threads are serer2 are shown in
shades of grey. On the very left of the trace, service1 can run uncontended on
any core when requests come in. Then service2 starts, and has a high load and
thus immediately starts running on all four cores, because it is Burstable. When
a request from service1 comes in, the thread processing that request has to
share the core equally with threads from service2, or other service1 threads,
while while service2 gets other cores all to itself.

This means that Burstable pods have a performance impact on the average and tail
latency of Guaranteed ones because both share cores where they both have threads
as equals, irrespective of who is running how much on other cores. In
particular, service1 has less load and thus less runnable threads. However, on
the few cores where its threads are runnable, it will have to share that core
equally; even while the server with higher load runs uncontended on other cores.
We conclude that the issue boils down to the locality of weights again: if the
core with both servers' threads knew service2 had other cores all to itself it
could compensate, but it doesn't know, so it treats both equally.

\subsection{\cgroups{} in isolation }

\fk{results from microbenchmark with 5 processes and 2 cores}

\subsection{Existing Linux mechanisms that aren't fixes}

\subsubsection{Load balancing}

Linux performs periodic load balancing, where it works to equalize the weight
across different cores. However, load balancing runs significiantly less
frequently than scheduling does, especially during high load. How often load
balancing runs in general is a complicated number that is dependent on how close
the two cores are in the CPU architecture hierarchy, as well as how loaded the
machine is.  \fk{load balancer balances load, not weights}
This means that in the context of a stable set of long-running
processes, load balancing can ensure that no two cores have wildly different
total weights and thus that the time each thread gets on each core reflects the
time it should get overall. However, in the context of short-lived request
processing with a constantly changing set of runnable worker threads, load
balancing is not enough.

\begin{figure}[t]
    \centering
    \includegraphics[width=\columnwidth]{graphs/srv-bg-weight-cmp-low.png}
    \caption{Changing the weight of the server beyond 100 has little impact on
    how much the weight 1 best effort task interferes with
    it}\label{fig:srv-bg-weight-cmp}
\end{figure}

\subsubsection{Bigger weight splits}

As we saw in our experiment, Kubernetes uses a weight split of around 79:1,
whereas Linux supports weights in the range of [10000,1]. We show that a larger
weight split does not improve the inrease in average or tail latency. We
demonstrate this with a microbenchmark, which runs a simple CPU-bound server
with a pool of worker threads. We run an open-loop client on a remote machine,
and then start two best effort workloads doing image resizing. We put the server
and the image resize job each in their own \cgroups{} group, and pin them to the
same set of four cores. The image resize job always has weight 1, and we vary
the weight of the server. \autoref{fig:srv-bg-weight-cmp} shows that using a
bigger weight split has no impact on the latency impact of the best effort task.
For all the weights, at a $\sim$85\% baseline utilization of the server the
server's mean latencies spike up from steady at around 6ms to $\sim$15ms, and
much higher for 99th percentile latencies. At a baseline utilization of 95\%,
those numbers increase to up to 40ms for the the mean latency, and 80ms for the
tail.

\subsubsection{Alternatives to \cgroups{}}

Weights remains the \cgroups{} interface of choice for Kubernetes as
well as others like Firecracker and libvirt, but Linux also supports
other mechanisms for scheduling.  The Linux scheduler is hierarchical:
it supports different \textit{scheduling classes}, within which there
might be different \textit{policies}. The \normalclass{} scheduling
class is the one most people know---it used to run a Completely Fair
Scheduler (CFS) and now runs a version of Earliest Eligible Virtual
Deadline First (EEVDF). It is the default scheduler, and \cgroups{}
weight is only effective within the \normalclass{} class.

\schedidle{} is Linux's one answer to doing a better job of separating best
effort tasks from those with reservations.
\schedidle{} is a scheduling policy that lives within the \normalclass{} class
alongside the default policy, which is \schednormal{}.\footnote{There is,
confusingly, also an Idle scheduling \textit{class}, but that is inaccessible to
userspace and exists solely to manage the core's transition in and out of being
actually idle (\ie{} running nothing).} 

Linux developers have improved \schedidle{} over the years, and it is meant to
support best effort workloads. It was also recently extended to have a
\cgroups{} interface file ~\cite{lkml-idle-cgroup}, so setting cpu.idle to
contain 1 puts the whole group under the \schedidle{} policy.

\begin{figure}[t]
    \centering
    \begin{subfigure}[t]{\columnwidth}
        \includegraphics[width=\columnwidth]{graphs/srv-bg-idle-low.png}
        \caption{\schedidle{} in low load (85\%)}\label{fig:srv-bg-idle-low}
        \vspace{12pt}
    \end{subfigure}
    \hspace{\fill}
    \begin{subfigure}[t]{\columnwidth}
        \includegraphics[width=\columnwidth]{graphs/srv-bg-idle-high.png}
        \caption{\schedidle{} in high load (95\%)}\label{fig:srv-bg-idle-high}
        \vspace{12pt}
    \end{subfigure}
    \vspace{4pt}
    \caption{\fk{XXX}}\label{fig:srv-bg-idle}
\end{figure}

To determine whether \schedidle{} solves the problem with \cgroups{}
weights, we run the same microbenchmark, using \schedidle{} for the
image resize job instead of using a low \cgroups{} weight;
\autoref{fig:srv-bg-idle} shows the results.  The \schedidle{} policy
for the \normalclass{} reduces the impact of the scheduling anomalies
but doesn't fix them.  In the appendix \autoref{s:alternatives} we also show that other
scheduling classes are not satisfactory alternatives to \cgroups{}.

\subsection{Summary}

From this section we conclude that Linux's \cgroups{} implementation
doesn't meet the specification stated in the documentation: each group
doesn't get time proportional to its weight as a share of the sum of
weights of runnable groups.  The reason Linux fails to achieve the
specification is because Linux uses per-CPU run-queues, and enforces
weights only per run-queue.  This implementation can result in
scheduling anomalies where one core doesn't realize that another core
has a runnable high-weight process, resulting in occasional latency
spikes for client requests to the high-weight process.






%-------------------------------------------------------------------------------
\section{Alternatives}\label{s:alternatives}
%-------------------------------------------------------------------------------

Weights remains the \cgroups{} interface of choice for most systems, but Linux
scheduler is a complicated system with many different parts. In this section we
discuss the existing alternatives to using \cgroups{} weights in Linux, and show
that none of them completely address the issue. 

The Linux scheduler is hierarchical: it supports different \textit{scheduling
classes}, which exist completely independently from one another: classes
maintain their own runqueues and per-entity state; implement their own
scheduling algorithms to choose from the entities on their runqueue; and balance
the load across runqueues on different cores.

The \normalclass{} scheduling class is the one most people know --- it used to
run a Completely Fair Scheduler (CFS) and now runs a version of Earliest
Eligible Virtual Deadline First (EEVDF). It is the default scheduler, and
\cgroups{} weight is only effective within the \normalclass{} class. The other
two scheduling classes that are available to users are \deadlineclass{} and
\rtclass{}, both of which are designed to support real-time applications.

\subsection{\schedidle{}}

\begin{figure}[t]
    \centering
    \begin{subfigure}[t]{\columnwidth}
        \includegraphics[width=\columnwidth]{graphs/srv-bg-idle-low.png}
        \caption{\schedidle{} in low load (85\%)}\label{fig:srv-bg-idle-low}
        \vspace{12pt}
    \end{subfigure}
    \hspace{\fill}
    \begin{subfigure}[t]{\columnwidth}
        \includegraphics[width=\columnwidth]{graphs/srv-bg-idle-high.png}
        \caption{\schedidle{} in high load (95\%)}\label{fig:srv-bg-idle-high}
        \vspace{12pt}
    \end{subfigure}
    \vspace{4pt}
    \caption{}\label{fig:srv-bg-idle}
\end{figure}

\schedidle{} is Linux's current answer to doing a better job of separating best
effort tasks from those with reservations.

Scheduling \textit{classes} in Linux can have multiple \textit{policies}, and
\schedidle{} lives within the \normalclass{} class alongside the default policy,
which is \schednormal{}.\footnote{There is, confusingly, also an Idle scheduling
\textit{class}, but that is inaccessible to userspace and exists solely to
manage the core's transition in and out of being actually idle (\ie{} running
nothing).} 

Linux developers have improved \schedidle{} over the years, and it is meant to
support best effort workloads. To determine whether \schedidle{} solves the
problem with \cgroups{} weights, we run the same microbenchmark, using
\schedidle{} for the image resize job instead of using a low \cgroups{} weight;
\autoref{fig:srv-bg-idle} shows the results.


\subsection{Other scheduling classes}


\begin{figure}[t]
    \centering
    \begin{subfigure}[t]{\columnwidth}
        \includegraphics[width=\columnwidth]{graphs/srv-bg-rt-low.png}
        \caption{Low load (85\%)}\label{fig:srv-bg-rt-low}
        \vspace{12pt}
    \end{subfigure}
    \hspace{\fill}
    \begin{subfigure}[t]{\columnwidth}
        \includegraphics[width=\columnwidth]{graphs/srv-bg-rt-high.png}
        \caption{High load (95\%)}\label{fig:srv-bg-rt-high}
    \end{subfigure}
    \vspace{4pt}
    \caption{when running the server as a real time application, Linux does a
     good job of isolating the server's latencies from the load from best effort
     jobs }\label{fig:srv-bg-rt}
\end{figure}

Given that the desired behavior is strong separation between best effort
processes and those with reserved resources, using separate scheduling classes
for the two is an attractive proposition. 

The \deadlineclass{} scheduling class is not a good fit for running
microservices, since it requires accurate knowledge of a processes runtime
(processing time per request) and period (when requests come in). These are
neither fixed nor known ahead of time in many applications. 

The \rtclass{} has no such requirements for processes running in it, and allows
for oversubscription. \autoref{fig:srv-bg-rt} shows the result of running the
microbenchmark, but with the server running in the \rtclass{} scheduling class.
We can see that in both load settings the tail and average latency stays stable
at $\sim$6.0ms after starting the BE workload. The throughput of the image
resize job (10 iter/100ms at low load, and 5iter at high load) is around 80\% of
what it was in \autoref{fig:srv-bg-weight} when running the server with
\cgroups{} weights.


However, we show that the schedulers within each priority are not suited for
microservices (\autoref{sss:approch:linux:policies}), and that Linux method of
avoiding starvation has adverse effects on the LC application
(\autoref{sss:approach:linux:starve-throttle}).

\subsubsection{Real time schedulers are unsuitable for microservice
workloads}\label{sss:approch:linux:policies}

Running a microservice in \rtclass{} is untenable because of \rtclass{}'s
intra-priority schedulers. \rtclass{} has two different scheduling policies:
\schedfifo{} and \schedrr{}. Both have 99 priorities between which they enforce
strict priority; within priorities they differ in the policy they enforce.

\schedfifo{} uses first-in first-out run-to-completion scheduling. This means
that within each class, the process to run next will be the one that woke up
first, and it will run until it blocks or exits. When a blocked thread becomes
runnable again, that is counted as a wakeup and it will be put on the back of
the queue. This is known to have a failure mode of head-of-line (HoL) blocking
under varied request processing times, where long-running requests monopolize
the CPU while short requests wait in the queue.

\schedrr{} addresses this concern by running a round-robin scheduler that will
ensure Processor Sharing within each priority. Every thread just gets the same
scheduling quantum and then gets put at the end of that priority's queue. This
means that within each priority CPU time is allocated based on the number of
runnable threads. This conflicts with how weights are used within runqueues to
allocate CPU time between different LC workloads. Kubernetes, for instance,
allows users to make fractional CPU requests, which are enforced using weights.
Although \cgroups{} limits (\ie{} defining a maximum amount of runtime per
period each group can get) can also be applied to groups in real time
applications, Kubernetes requires the weight interface to allow for Burstable
pods.

\subsubsection{Linux throttles \rtclass{} processes under high load
}\label{sss:approach:linux:starve-throttle}

Schedulers running priority scheduling have to contend with the possibility of
starvation. Starvation can have many negative effects: it can cause deadlocks if
a low and high priority process share a lock (either in user-space or in
kernel-space), TCP connections can die while the process is being starved, and
it can miss interrupts like timers or completed i/o requests.

To avoid these, Linux chooses to ensure that no process is ever starved, which
it does by throttling high class processes with high load. Linux has two
different safeguards that enforce that no process is ever starved. One is that
\rtclass{} is as a scheduling class rate-limited: there are tuneable parameters
\texttt{sched\_rt\_runtime} and \texttt{sched\_rt\_period}, that together define
a rate limit for the \rtclass{} as a whole. The other safegaurd is that, even
when set to be equal (\ie{} \rtclass{} gets the full runtime each period if it
wants), the \normalclass{} scheduling class also has a so-called
\textit{deadline server}, which ensures it gets a small amount of time. The
deadline server is a `process' is in the \deadlineclass{} scheduling class with
a small amount of runtime per period, then when chosen will pass control on to
the \normalclass{} scheduler~\cite{lkml-deadline-srv}.

The throttling Linux chooses to do interferes with the goal of honoring
reservations, because it throttles the LC experiencing high load, which is
precisly when it needs its full reservation  most.


\begin{figure}[t]
    \centering
    \includegraphics[width=\columnwidth]{graphs/overload-rt.png}
    \caption{LC in real time, throttling}\label{fig:overload-rt}
\end{figure}

Doing so impacts the performance of processes in the \rtclass{} at high load. We
can see this happen when we run the same microbenchmark experiment at a much
higher baseline utilization ($\sim$ 100\%). The results are in
\autoref{fig:overload-rt}. We see spikes begin to appear after starting the
image resize job, as the \rtclass{} server gets throttled in favor of running
the BE task; we see parallel spikes in the BE's throughput in the bottom graph.
Notice also the increase of the slope of response times after starting the
background tasks, this happens as the client has to queue requests while all the
current connections are blocked on running requests.

We conclude that Linux's mechanism of scheduling classes can enforce
reservations effectively, but that existing scheduling classes use algorithms
that are not a good fit for modern workloads, and that Linux doesn't enforce
reservations under high load because it throttles high priority classes.






%\section{Approach \& Uniqueness}

In order to enforce reservations while still running best effort jobs
opportunistically, our approach is use priority scheduling.

Enforcing priorities requires fewer global runqueue searches than weights do,
because they only need to happen on \textit{class boundary crossings}: on
\exit{}, when a core switches to running lower class processes after having
previously been running high class, and on \entry{}, when a core enqueues a
higher class process. These checks ensure that a core $c$ running a BE thread
$t$ knows that there are no queued LC threads anywhere on the machine. If there
was one when $c$ starts running $t$, the \exit{} check would see and steal it.
If a new LC thread wakes up on a different core while $t$ is running, the
\entry{} check ensures that core will look at $c$'s runqueue and run the new LC
thead on $c$, interrupting $t$.

\begin{figure}[t]
    \centering
    \includegraphics[width=\columnwidth]{graphs/overload-rt.png}
    \caption{LC in real time, throttling}\label{fig:overload-rt}
\end{figure}


Linux already provides priority scheduling across scheduling classes, which are
used to separate real-time applications from all other workloads. However, the
scheduling algorithm used for real-time applications differs from the one for
the default class. Additionally, the priority between real-time and the default
scheduling classes comes with an asterisk: if real-time applications experience
high load, Linux throttles them in order to not starve the default class. We can
see this happening in \autoref{fig:overload-rt}, where throttling leads to
spikes in the background task as it is able to run, and corresponding spikes in
the server's latency.

We design a new scheduling class \beclass{} that sits at a lower priority than
\normalclass{}, which enforces LC applications' uncontended access to the CPUs
they reserved. To enforce reservations even under high load, without throttling
the latency critical services or killing the BE ones, \beclass{} \textit{parks}
BE processes meaning the user-space code never runs, only kernel-level
services.
%\section{Design}\label{design}


\subsection{Approach}

Our approach is centered around \priceclass{}es. \Priceclass{}es are a metric
that has a number of benefits over resource reservations as an interface:
developers are more likely to have a good sense of it ahead of time, it is less
likely to be different across different invocations, it still gives the
scheduler the information it needs to decide what to schedule when, and finally
it more directly aligns the interests of the developer with those of the
provider by communicating on the level of what the provider and developer
actually care about: money, and latency (as achieved by \class{}es in the
system).


However, having \priceclass{}es also means that there are no clear guarantees
about what they are getting when a developer puts a price on a function they
want to run. In order to mitigate that somewhat and not go into bidding wars, we
propose exposing a fixed set of price classes. This is similar to how AWS has
different EC2 instance types, that are directly mapped to prices. Rather than
being a guarantee, the price class is instead a metric to express priority to
\sys{}, which it can then use to enforce a favoring of high class jobs.



\subsection{Interface}


Developers using \sys{} write function handlers and define triggers just like
they would for any existing serverless offering. In addition, they asign each
function to a price class, this is done at function creation. For instance, a
simple web server might consist of a home page view that is assigned a higher
\priceclass{} and costs 2$\mu\cent$ per cpu second, a user profile page view
which is assigned a middle-high \class{} and cost 1.5$\mu\cent$ per cpu second,
and finally an image processing job that can be set to a low \class{} which
costs only 0.5$\mu\cent$ per cpu second.

\Class{}es are inherited across call chains: if a high \class{} job calls a low
\class{} job, that invocation with run with high \class{}. This is important in
order to avoid priority inversion.

Developers pay for memory separately, and by use; the price for memory is the
same across all \class{}es.

To avoid unexpected costs in the case of for example a DOS attack or a bug,
developers also express a monthly budget that they are willing to pay.\ \sys{}
uses this budget as a guideline and throttles invocations or decreases quality
of service in the case that usage is not within reason given the expected
budget, though it does not guarantee that the budget will not be exceeded by
small amounts.



\subsection{\Sys{} Design}

\begin{figure}[t]
    \centering
      \includegraphics[width=9cm]{img/overview.png}
      \caption{ global scheduler shards queue and place jobs (in orange), 
      on each machine a dispatcher thread keeps track of memory utilization 
      and if it's low writes itself to an idle list (in blue) }
    \label{fig:overview}
\end{figure}



\Sys{} has as its goal to enforce the \class{}es attached to jobs, which means
it needs to prefer higher \class{} jobs over lower ones, and preempt the latter
when necessary.
  

As shown in Figure~\ref{fig:overview}, \sys{} sits behind a load balancer, and
consists of: a \textit{distributed global scheduler}, which places new job
invocations and has attached an \textit{idle list}, a \textit{dispatcher},
which runs on each machine and communicates with the global scheduler shards,
and a \textit{machine scheduler}, which enforces \class{}es on the machines.


\textbf{Machine Scheduler.}
The machine scheduler is a preemptive priority scheduler: it preempts lower
\class{} jobs to run higher \class{} ones. Being unfair and starving low
\class{} jobs is desirable in \sys{}, since image processing jobs should not
interrupt a page view request processing, but vice versa is expected. Within
\class{}es the machine scheduler is first come first served. This matches Linux'
`sched fifo' scheduling.
% https://man7.org/linux/man-pages/man7/sched.7.html


\textbf{Idle list.}
Each global scheduler shard has an idle list, which holds machines that
have a significant amount of memory available. In the shards idle list each
machine's entry is associated with the amount of free memory as well as the
current amount of jobs on the machine. The idle list exists because datacenters
are large: polling a small number of machines has been shown to be very
powerful, but cannot find something that is a very rare occurrence.
% join idle queue
What this means is that polling is likely to find a machine in the lower
quantile of the datacenter, but not at the absolute bottom --- it will not find
one of the handful of machines that are actually idle. Having an idle list
allows these machines, which are expected to be rare in a high-utilization
setting, to make themselves visible to the global scheduler. The idle list also
allows the global scheduler to place high \class{} processes quickly, without
incurring the latency overheads of doing the polling to find available
resources.


\textbf{Dispatcher.}
The dispatcher is in charge of adding itself to a shard's idle list when memory
utilization is low. The dispatcher chooses which list to add itself to using
power-of-k-choices: it looks at k shards' idle lists and chooses the one with
the least other machines in it. If the machine is already on an idle list on
shard $i$, but the amount of available memory has changed significantly (either
by jobs finishing and memory being freed or by memory utilization increasing
because of new jobs or memory antagonists), the dispatcher will update shard
$i$'s idle list. These interactions from the dispatcher to free lists are
represented by the blue arrows in Figure~\ref{fig:overview}.

The dispatcher is also in charge of managing the machine's memory. When memory
pressure occurs, the dispatcher uses \textit{\class{}-based swapping} to move low
\class{} processes off the machine's memory. In this scenario, having priority
scheduling creates the opportunity that enables this to be realistic: because
the dispatcher knows that the lowest \class{} jobs will not be run until the
high \class{} jobs have all finished, it can swap its memory out knowing it will
not be needed soon. Avoiding a churn of jobs with swapped memory that are being
swapped in and out as they are scheduled in and out, requires that the memory of
the machine is large against the amount of memory that could be used by as many
jobs as the machine has cores. This assumption ensures that memory pressure high
enough to require swapping will only occur when there are many more jobs than
there are cores, and thus that the swapped job will not be running soon. The
dispatcher swaps the low \class{} job back in when the memory pressure is gone
and the job will be run.

% \makeatletter
% \renewcommand{\ALG@name}{Procedure}
% \makeatother
% \begin{algorithm}[t]
% \caption{Choosing a machine for a job j}\label{alg:place}
% \begin{algorithmic}
%     \If{$|idleList| > 0$} \\
%         \Return$ $min(qLen)
%     \EndIf
%     \State$M = $ qlen of k polled machines
%     \If{min(M.timeToProfit$) < THRESH$} \\
%         \Return$ $min($M$)
%     \Else
%         \State$ $reQ j, with priority -= 1
%     \EndIf
% \end{algorithmic}
% \end{algorithm}


\textbf{Global Scheduler Shards.}

Global scheduler shards store the jobs waiting to be placed in a multi queue,
with one queue per \priceclass{}. Shards choose what job to place next by
looking at each job at the head of each queue, and comparing the ratio of
\class{} to amount of time spent in the queue. This ensures that high \class{}
jobs don't have to wait as long as lower \class{} jobs to be chosen next, but
low \class{} jobs will get placed if they have waited for a while.

When placing the chosen job, the shard will first look in its idle list. If the
list is not empty, it will choose the machine with the smallest queue length.

If there are no machines in the idle list, the shard switches over to
power-of-k-choices: it polls k machines, getting the amount of jobs running from
each. The shard then places the new job on the machine with the smallest number
of currently running jobs. It is desirable to have a maximally heterogenous set
of \class{}es on each machine, but since jobs come in randomly the global
scheduler does not explicitly have to enforce this.

\section{Design of \sys}
\label{sec:global}

\sys is a new design for scheduling processes that enforces group
weights globally across cores.  It assigns each runnable process a
global virtual time based on its group weight. \sys maintains a heap
of these processes based on their virtual runtimes, with the process
with lowest virtual time at the front of the heap
(\autoref{sec:vt}). To implement this heap in a scalable manner, \sys
uses an {\it relaxed} global heap and may select a process that has a
virtual time higher than the global minimum virtual time
(\autoref{sec:relaxed}).  Finally, if several processes of the same
group are runnable, \sys selects the one that it has run recently to
provide core affinity so that the selected process runs with a warm
L1/L2 caches.

\subsection{Global virtual time}
\label{sec:vt}

\begin{figure}
\begin{Verbatim}[commandchars=\\\{\},numbers=left,firstnumber=1,stepnumber=1,codes={\catcode`\$=3\catcode`\^=7\catcode`\_=8\relax},fontsize=\scriptsize,numbersep=6pt,xleftmargin=0.2in]
\PY{k}{struct}\PY{+w}{ }\PY{n+nc}{group}\PY{+w}{ }\PY{p}{\PYZob{}}
\PY{+w}{  }\PY{n}{w\PYZus{}t}\PY{+w}{ }\PY{n}{weight}\PY{p}{;}
\PY{+w}{  }\PY{n}{vt\PYZus{}t}\PY{+w}{ }\PY{n}{vruntime}\PY{p}{;}
\PY{+w}{  }\PY{n}{vt\PYZus{}t}\PY{+w}{ }\PY{n}{min\PYZus{}vt\PYZus{}deq}\PY{p}{;}
\PY{p}{\PYZcb{}}
\end{Verbatim}

\caption{The state for each group}
\label{fig:group}
\end{figure}

\autoref{fig:group} shows the state that \sys maintains for each
cgroup: \cc{weight}, the groups weight, \cc{vruntime}, which is the
next available virtual runtime for a process in the cgroup to run, and
\cc{min_vt_deq}, which is the virtual runtime when a group goes to
  ``sleep'' (i.e., when all group's processes are sleeping).

\begin{figure}
\begin{Verbatim}[commandchars=\\\{\},numbers=left,firstnumber=1,stepnumber=1,codes={\catcode`\$=3\catcode`\^=7\catcode`\_=8\relax},fontsize=\scriptsize,numbersep=6pt,xleftmargin=0.2in]
\PY{n}{vt\PYZus{}t}\PY{+w}{ }\PY{n+nf}{calc\PYZus{}delta}\PY{p}{(}\PY{n}{vt\PYZus{}t}\PY{+w}{ }\PY{n}{delta}\PY{p}{,}\PY{+w}{ }\PY{n}{w\PYZus{}t}\PY{+w}{ }\PY{n}{weight}\PY{p}{)}\PY{+w}{ }\PY{p}{\PYZob{}}
\PY{+w}{        }\PY{k}{return}\PY{+w}{ }\PY{n}{delta}\PY{+w}{ }\PY{o}{/}\PY{+w}{ }\PY{n}{weight}\PY{p}{;}
\PY{p}{\PYZcb{}}

\PY{k+kt}{void}\PY{+w}{ }\PY{n+nf}{enq\PYZus{}proc\PYZus{}vt}\PY{p}{(}\PY{k}{struct}\PY{+w}{ }\PY{n+nc}{global\PYZus{}heap}\PY{+w}{ }\PY{o}{*}\PY{n}{gh}\PY{p}{,}\PY{+w}{ }\PY{k}{struct}\PY{+w}{ }\PY{n+nc}{process}\PY{+w}{ }\PY{o}{*}\PY{n}{p}\PY{p}{,}\PY{+w}{ }\PY{k}{struct}\PY{+w}{ }\PY{n+nc}{heap}\PY{+w}{ }\PY{o}{*}\PY{n}{h}\PY{p}{)}\PY{+w}{ }\PY{p}{\PYZob{}}
\PY{+w}{  }\PY{n}{vt\PYZus{}t}\PY{+w}{ }\PY{n}{wvt}\PY{+w}{ }\PY{o}{=}\PY{+w}{ }\PY{n}{calc\PYZus{}delta}\PY{p}{(}\PY{n}{gh}\PY{o}{\PYZhy{}}\PY{o}{\PYZgt{}}\PY{n}{tick\PYZus{}length}\PY{p}{,}\PY{+w}{ }\PY{n}{p}\PY{o}{\PYZhy{}}\PY{o}{\PYZgt{}}\PY{n}{he}\PY{p}{.}\PY{n}{weight}\PY{p}{)}\PY{p}{;}
\PY{+w}{  }\PY{n}{vt\PYZus{}t}\PY{+w}{ }\PY{n}{my\PYZus{}vt}\PY{+w}{ }\PY{o}{=}\PY{+w}{ }\PY{n}{atomic\PYZus{}fetch\PYZus{}add}\PY{p}{(}\PY{o}{\PYZam{}}\PY{n}{p}\PY{o}{\PYZhy{}}\PY{o}{\PYZgt{}}\PY{n}{group}\PY{o}{\PYZhy{}}\PY{o}{\PYZgt{}}\PY{n}{vruntime}\PY{p}{,}\PY{+w}{ }\PY{n}{vt}\PY{p}{)}\PY{p}{;}
\PY{+w}{  }\PY{n}{p}\PY{o}{\PYZhy{}}\PY{o}{\PYZgt{}}\PY{n}{he}\PY{p}{.}\PY{n}{vruntime}\PY{+w}{ }\PY{o}{=}\PY{+w}{ }\PY{n}{my\PYZus{}vt}\PY{p}{;}
\PY{+w}{  }\PY{n}{mh\PYZus{}add\PYZus{}process}\PY{p}{(}\PY{n}{p}\PY{p}{,}\PY{+w}{ }\PY{n}{h}\PY{p}{)}\PY{p}{;}
\PY{p}{\PYZcb{}}
\end{Verbatim}

\caption{Enqueue a process}
\label{fig:enq}
\end{figure}

When a process of in a group becomes runnable, \sys assigns it a
virtual time and inserts it in the global heap, as show in
\autoref{fig:enq}. Consider a \cc{tick_length} of 1000us and two
groups, $g_1$ with weight 10 and $g_2$ with weight 20. When a process
of $g_1$ becomes runnable it will get assigned the current group's
virtual time and increases it by 100 ($= 1000/10$).  If concurrently,
on another core, another process of $g_1$ becomes runnable it will be
assigned a virtual time that is 100 larger than the first process, and
$g_1$'s virtual time will be incremented by 100, ensuring the
processes are correctly spaced in virtual time by 100.

When a process of $g_2$ becomes runnable, it will increases its
group's virtual time by 50 ($= 1000/20$), and the processes of $g_2$
will be spaced in virtual time by 50.  If we look over a window of a
1000 in virtual time, $g_1$ will run 10 times and $g_2$ will run 20
times, reflecting their relative weights of 10 and 20.

\begin{figure}
\begin{Verbatim}[commandchars=\\\{\},numbers=left,firstnumber=1,stepnumber=1,codes={\catcode`\$=3\catcode`\^=7\catcode`\_=8\relax},fontsize=\scriptsize,numbersep=6pt,xleftmargin=0.2in]
\PY{k+kt}{void}\PY{+w}{ }\PY{n+nf}{grp\PYZus{}adjust\PYZus{}vruntime}\PY{p}{(}\PY{k}{struct}\PY{+w}{ }\PY{n+nc}{global\PYZus{}heap}\PY{+w}{ }\PY{o}{*}\PY{n}{gh}\PY{p}{,}\PY{+w}{ }\PY{k}{struct}\PY{+w}{ }\PY{n+nc}{process}\PY{+w}{ }\PY{o}{*}\PY{n}{p}\PY{p}{,}\PY{+w}{ }\PY{n}{t\PYZus{}t}\PY{+w}{ }\PY{n}{time\PYZus{}passed}\PY{p}{)}\PY{+w}{ }\PY{p}{\PYZob{}}
\PY{+w}{  }\PY{n}{vt\PYZus{}t}\PY{+w}{ }\PY{n}{vt}\PY{+w}{ }\PY{o}{=}\PY{+w}{ }\PY{n}{calc\PYZus{}delta}\PY{p}{(}\PY{n}{time\PYZus{}passed}\PY{p}{,}\PY{+w}{ }\PY{n}{p}\PY{o}{\PYZhy{}}\PY{o}{\PYZgt{}}\PY{n}{he}\PY{p}{.}\PY{n}{weight}\PY{p}{)}\PY{p}{;}
\PY{+w}{  }\PY{n}{vt\PYZus{}t}\PY{+w}{ }\PY{n}{wvt}\PY{+w}{ }\PY{o}{=}\PY{+w}{ }\PY{n}{calc\PYZus{}delta}\PY{p}{(}\PY{n}{gh}\PY{o}{\PYZhy{}}\PY{o}{\PYZgt{}}\PY{n}{tick\PYZus{}length}\PY{p}{,}\PY{+w}{ }\PY{n}{p}\PY{o}{\PYZhy{}}\PY{o}{\PYZgt{}}\PY{n}{he}\PY{p}{.}\PY{n}{weight}\PY{p}{)}\PY{p}{;}
\PY{+w}{  }\PY{k}{if}\PY{+w}{ }\PY{p}{(}\PY{n}{wvt}\PY{+w}{ }\PY{o}{\PYZgt{}}\PY{+w}{ }\PY{n}{vt}\PY{p}{)}\PY{+w}{ }\PY{p}{\PYZob{}}
\PY{+w}{    }\PY{n}{atomic\PYZus{}fetch\PYZus{}add}\PY{p}{(}\PY{o}{\PYZam{}}\PY{n}{p}\PY{o}{\PYZhy{}}\PY{o}{\PYZgt{}}\PY{n}{group}\PY{o}{\PYZhy{}}\PY{o}{\PYZgt{}}\PY{n}{vruntime}\PY{p}{,}\PY{+w}{ }\PY{o}{\PYZhy{}}\PY{p}{(}\PY{n}{wvt}\PY{o}{\PYZhy{}}\PY{n}{vt}\PY{p}{)}\PY{p}{)}\PY{p}{;}
\PY{+w}{  }\PY{p}{\PYZcb{}}
\PY{p}{\PYZcb{}}
\end{Verbatim}

\caption{Adjust virtual runtime when a process yield before a full clock tick.}
\label{fig:adjust}
\end{figure}

If a process yields its core before \cc{tick_length}, \sys adjusts the
group's virtual time to reflect that the process didn't use the full
tick, as shown in \autoref{fig:adjust}.  The adjustment moves move the
group's virtual time backwards so that the next process of the group
inherits the time that the first process didn't use.

\begin{figure}
\begin{Verbatim}[commandchars=\\\{\},numbers=left,firstnumber=1,stepnumber=1,codes={\catcode`\$=3\catcode`\^=7\catcode`\_=8\relax},fontsize=\scriptsize,numbersep=6pt,xleftmargin=0.2in]
\PY{k+kt}{void}\PY{+w}{ }\PY{n+nf}{set\PYZus{}vruntime}\PY{p}{(}\PY{k}{struct}\PY{+w}{ }\PY{n+nc}{process}\PY{+w}{ }\PY{o}{*}\PY{n}{p}\PY{p}{)}\PY{+w}{ }\PY{p}{\PYZob{}}
\PY{+w}{  }\PY{n}{vt\PYZus{}t}\PY{+w}{ }\PY{n}{lag}\PY{+w}{ }\PY{o}{=}\PY{+w}{ }\PY{n}{p}\PY{o}{\PYZhy{}}\PY{o}{\PYZgt{}}\PY{n}{group}\PY{o}{\PYZhy{}}\PY{o}{\PYZgt{}}\PY{n}{vruntime}\PY{+w}{ }\PY{o}{\PYZhy{}}\PY{+w}{ }\PY{n}{p}\PY{o}{\PYZhy{}}\PY{o}{\PYZgt{}}\PY{n}{group}\PY{o}{\PYZhy{}}\PY{o}{\PYZgt{}}\PY{n}{min\PYZus{}vt\PYZus{}deq}\PY{p}{;}
\PY{+w}{  }\PY{n}{vt\PYZus{}t}\PY{+w}{ }\PY{n}{h\PYZus{}min}\PY{+w}{ }\PY{o}{=}\PY{+w}{ }\PY{n}{mh\PYZus{}min\PYZus{}vt}\PY{p}{(}\PY{n}{h}\PY{p}{)}\PY{p}{;}
\PY{+w}{  }\PY{k}{if}\PY{p}{(}\PY{n}{p}\PY{o}{\PYZhy{}}\PY{o}{\PYZgt{}}\PY{n}{group}\PY{o}{\PYZhy{}}\PY{o}{\PYZgt{}}\PY{n}{min\PYZus{}vt\PYZus{}deq}\PY{+w}{ }\PY{o}{\PYZgt{}}\PY{+w}{ }\PY{n}{h\PYZus{}min}\PY{p}{)}\PY{+w}{ }\PY{p}{\PYZob{}}
\PY{+w}{    }\PY{n}{lag}\PY{+w}{ }\PY{o}{+}\PY{o}{=}\PY{+w}{ }\PY{p}{(}\PY{n}{p}\PY{o}{\PYZhy{}}\PY{o}{\PYZgt{}}\PY{n}{group}\PY{o}{\PYZhy{}}\PY{o}{\PYZgt{}}\PY{n}{min\PYZus{}vt\PYZus{}deq}\PY{o}{\PYZhy{}}\PY{n}{h\PYZus{}min}\PY{p}{)}\PY{p}{;}
\PY{+w}{  }\PY{p}{\PYZcb{}}
\PY{+w}{  }\PY{n}{atomic\PYZus{}store}\PY{p}{(}\PY{o}{\PYZam{}}\PY{n}{p}\PY{o}{\PYZhy{}}\PY{o}{\PYZgt{}}\PY{n}{group}\PY{o}{\PYZhy{}}\PY{o}{\PYZgt{}}\PY{n}{vruntime}\PY{p}{,}\PY{+w}{ }\PY{n}{h\PYZus{}min}\PY{o}{+}\PY{n}{lag}\PY{p}{)}\PY{p}{;}
\PY{p}{\PYZcb{}}
\end{Verbatim}

\caption{Set group's virtual time when it becomes runnable.}
\label{fig:lag}
\end{figure}

When the last running process of a group goes to sleep, \sys remembers
the minimum virtual time of the heap in the process group's
\cc{min_vt_deq}. It does so to set the group's \cc{vruntime} when
the group becomes runnable again, as shown in \autoref{fig:lag}.  When
the group becomes runnable, the global virtual runtime may have moved
ahead since the group went to sleep. To account for this, \sys sets
the group's virtual time to the current minimum global virtual time,
adjusted for how much the group was ahead or behind the global minimum
when the group dequeued.

The way \sys computes virtual time is similar to Linux's fair-share
scheduler with the main difference being that \sys must compute the
virtual time globally, and, handle multiple cores concurrently
assigning virtual times to processes of the same group.  In Linux
virtual time is per run queue (and not global), and updated only by
one core. As a result, Linux, for example, doesn't need to update the
group's virtual on enqueue, doesn't need to adjust it if a process
runs shorter than its full tick, and doesn't need to set a group's
virtual time when it wakes up.

\subsection{\mheap: an relaxed global heap}
\label{sec:relaxed}

\sys's heap of runnable processes is often written by many cores
concurrently: cores remove the process with the minimal virtual time
from the heap and insert runnable processes in the heap based on their
virtual times.  To avoid operations contending on the global heap, \sys
uses a multi-heap design with a relaxed global minimum, which is based
on relaxed concurrent priority queues~\cite{multiqueue:2015}.

In concurrent priority queues, a core inserts an element in a random
queue out of $n \times d$ queues, where $n$ is the number of cores and
$d$ is a small constant (typically 2).  To find the minimum element to delete,
concurrent priority queues chooses two random queues and removes the
smallest one. The intuition behind this idea is from randomized load
balancing: choosing two randomly-chosen machines and add a ball to the
least-loaded one results in a maximum load very close to the average
load with high
probability~\cite{mitzenmacher:power,berenbrink:balanced}.

Concurrent priority queues are attractive for implementing \sys
because they allow for concurrent write operations.  Each queue has
its own lock and cores retry choosing a random queue if the chosen
queue is locked by another core.  Williams et al. show---in theory and
in practice---that concurrent priority queues can achieve throughout
that scales with increasing number of cores and that the rank error (the
distance of the deleted element to the best element) is in expectation
$O(c \log c)$ with high probability~\cite{williams:multiqueue}.

\begin{figure}
\begin{Verbatim}[commandchars=\\\{\},numbers=left,firstnumber=1,stepnumber=1,codes={\catcode`\$=3\catcode`\^=7\catcode`\_=8\relax},fontsize=\scriptsize,numbersep=6pt,xleftmargin=0.2in]
\PY{k}{struct}\PY{+w}{ }\PY{n+nc}{process}\PY{+w}{ }\PY{o}{*}\PY{n}{min\PYZus{}affinity}\PY{p}{(}\PY{k}{struct}\PY{+w}{ }\PY{n+nc}{core}\PY{+w}{ }\PY{o}{*}\PY{n}{c}\PY{p}{)}\PY{+w}{ }\PY{p}{\PYZob{}}
\PY{+w}{  }\PY{k}{struct}\PY{+w}{ }\PY{n+nc}{process}\PY{+w}{ }\PY{o}{*}\PY{n}{cp}\PY{+w}{ }\PY{o}{=}\PY{+w}{ }\PY{n}{c}\PY{o}{\PYZhy{}}\PY{o}{\PYZgt{}}\PY{n}{process}\PY{p}{;}
\PY{+w}{  }\PY{k}{struct}\PY{+w}{ }\PY{n+nc}{heap}\PY{+w}{ }\PY{o}{*}\PY{n}{h}\PY{+w}{ }\PY{o}{=}\PY{+w}{ }\PY{n}{cp}\PY{o}{\PYZhy{}}\PY{o}{\PYZgt{}}\PY{n}{h}\PY{p}{;}
\PY{+w}{  }\PY{k+kt}{int}\PY{+w}{ }\PY{n}{cid}\PY{+w}{ }\PY{o}{=}\PY{+w}{ }\PY{n}{atomic\PYZus{}load}\PY{p}{(}\PY{o}{\PYZam{}}\PY{n}{cp}\PY{o}{\PYZhy{}}\PY{o}{\PYZgt{}}\PY{n}{cid}\PY{p}{)}\PY{p}{;}
\PY{+w}{  }\PY{k}{if}\PY{+w}{ }\PY{p}{(}\PY{n}{cid}\PY{+w}{ }\PY{o}{!}\PY{o}{=}\PY{+w}{ }\PY{n}{c}\PY{o}{\PYZhy{}}\PY{o}{\PYZgt{}}\PY{n}{cid}\PY{p}{)}\PY{+w}{ }\PY{p}{\PYZob{}}
\PY{+w}{    }\PY{k}{return}\PY{+w}{ }\PY{n+nb}{NULL}\PY{p}{;}
\PY{+w}{  }\PY{p}{\PYZcb{}}
\PY{+w}{  }\PY{k}{if}\PY{p}{(}\PY{n}{atomic\PYZus{}load}\PY{p}{(}\PY{o}{\PYZam{}}\PY{n}{h}\PY{o}{\PYZhy{}}\PY{o}{\PYZgt{}}\PY{n}{heap}\PY{p}{[}\PY{l+m+mi}{0}\PY{p}{]}\PY{p}{.}\PY{n}{elem}\PY{p}{)}\PY{+w}{ }\PY{o}{!}\PY{o}{=}\PY{+w}{ }\PY{n}{cp}\PY{p}{)}\PY{+w}{ }\PY{p}{\PYZob{}}
\PY{+w}{    }\PY{k}{return}\PY{+w}{ }\PY{n+nb}{NULL}\PY{p}{;}
\PY{+w}{  }\PY{p}{\PYZcb{}}
\PY{+w}{  }\PY{k}{struct}\PY{+w}{ }\PY{n+nc}{process}\PY{+w}{ }\PY{o}{*}\PY{n}{p}\PY{+w}{ }\PY{o}{=}\PY{+w}{ }\PY{n+nb}{NULL}\PY{p}{;}
\PY{+w}{  }\PY{n}{lock\PYZus{}acquire}\PY{p}{(}\PY{o}{\PYZam{}}\PY{n}{h}\PY{o}{\PYZhy{}}\PY{o}{\PYZgt{}}\PY{n}{lk}\PY{p}{)}\PY{p}{;}
\PY{+w}{  }\PY{k}{if}\PY{p}{(}\PY{p}{(}\PY{n}{h}\PY{o}{\PYZhy{}}\PY{o}{\PYZgt{}}\PY{n}{heap}\PY{p}{[}\PY{l+m+mi}{0}\PY{p}{]}\PY{p}{.}\PY{n}{elem}\PY{+w}{ }\PY{o}{!}\PY{o}{=}\PY{+w}{ }\PY{n}{cp}\PY{p}{)}\PY{+w}{ }\PY{o}{|}\PY{o}{|}\PY{+w}{ }\PY{p}{(}\PY{n}{cp}\PY{o}{\PYZhy{}}\PY{o}{\PYZgt{}}\PY{n}{cid}\PY{+w}{ }\PY{o}{!}\PY{o}{=}\PY{+w}{ }\PY{n}{c}\PY{o}{\PYZhy{}}\PY{o}{\PYZgt{}}\PY{n}{cid}\PY{p}{)}\PY{p}{)}\PY{+w}{ }\PY{p}{\PYZob{}}
\PY{+w}{    }\PY{k}{goto}\PY{+w}{ }\PY{n}{end}\PY{p}{;}
\PY{+w}{  }\PY{p}{\PYZcb{}}
\PY{n+nl}{retry}\PY{p}{:}
\PY{+w}{  }\PY{n}{vt\PYZus{}t}\PY{+w}{ }\PY{n}{vt}\PY{p}{;}
\PY{+w}{  }\PY{k+kt}{int}\PY{+w}{ }\PY{n}{j}\PY{+w}{ }\PY{o}{=}\PY{+w}{ }\PY{n}{mh\PYZus{}rand\PYZus{}heap}\PY{p}{(}\PY{n}{cp}\PY{o}{\PYZhy{}}\PY{o}{\PYZgt{}}\PY{n}{mh}\PY{p}{,}\PY{+w}{ }\PY{n}{c}\PY{p}{,}\PY{+w}{ }\PY{n}{h}\PY{o}{\PYZhy{}}\PY{o}{\PYZgt{}}\PY{n}{id}\PY{p}{)}\PY{p}{;}
\PY{+w}{  }\PY{k}{struct}\PY{+w}{ }\PY{n+nc}{heap}\PY{+w}{ }\PY{o}{*}\PY{n}{h1}\PY{+w}{ }\PY{o}{=}\PY{+w}{ }\PY{n}{select\PYZus{}affinity}\PY{p}{(}\PY{n}{h}\PY{o}{\PYZhy{}}\PY{o}{\PYZgt{}}\PY{n}{id}\PY{p}{,}\PY{+w}{ }\PY{n}{j}\PY{p}{,}\PY{+w}{ }\PY{o}{\PYZam{}}\PY{n}{vt}\PY{p}{)}\PY{p}{;}
\PY{+w}{  }\PY{k}{if}\PY{+w}{ }\PY{p}{(}\PY{n}{h1}\PY{+w}{ }\PY{o}{=}\PY{o}{=}\PY{+w}{ }\PY{n}{h}\PY{p}{)}\PY{+w}{ }\PY{p}{\PYZob{}}
\PY{+w}{    }\PY{n}{p}\PY{+w}{ }\PY{o}{=}\PY{+w}{ }\PY{n}{remove\PYZus{}min}\PY{p}{(}\PY{n}{h}\PY{p}{)}\PY{p}{;}
\PY{+w}{    }\PY{k}{goto}\PY{+w}{ }\PY{n}{end}\PY{p}{;}
\PY{+w}{  }\PY{p}{\PYZcb{}}
\PY{+w}{  }\PY{k+kt}{int}\PY{+w}{ }\PY{n}{l}\PY{+w}{ }\PY{o}{=}\PY{+w}{ }\PY{n}{lock\PYZus{}try\PYZus{}acquire}\PY{p}{(}\PY{o}{\PYZam{}}\PY{n}{h1}\PY{o}{\PYZhy{}}\PY{o}{\PYZgt{}}\PY{n}{lk}\PY{p}{)}\PY{p}{;}
\PY{+w}{  }\PY{k}{if}\PY{+w}{ }\PY{p}{(}\PY{n}{l}\PY{+w}{ }\PY{o}{!}\PY{o}{=}\PY{+w}{ }\PY{l+m+mi}{0}\PY{p}{)}\PY{+w}{ }\PY{p}{\PYZob{}}
\PY{+w}{    }\PY{k}{goto}\PY{+w}{ }\PY{n}{retry}\PY{p}{;}
\PY{+w}{  }\PY{p}{\PYZcb{}}
\PY{+w}{  }\PY{n}{vt\PYZus{}t}\PY{+w}{ }\PY{n}{vt0}\PY{+w}{ }\PY{o}{=}\PY{+w}{ }\PY{n}{h1}\PY{o}{\PYZhy{}}\PY{o}{\PYZgt{}}\PY{n}{heap}\PY{p}{[}\PY{l+m+mi}{0}\PY{p}{]}\PY{p}{.}\PY{n}{vruntime}\PY{p}{;}
\PY{+w}{  }\PY{k}{if}\PY{+w}{ }\PY{p}{(}\PY{n}{vt}\PY{+w}{ }\PY{o}{!}\PY{o}{=}\PY{+w}{ }\PY{n}{vt0}\PY{p}{)}\PY{+w}{ }\PY{p}{\PYZob{}}
\PY{+w}{    }\PY{n}{lock\PYZus{}release}\PY{p}{(}\PY{o}{\PYZam{}}\PY{n}{h1}\PY{o}{\PYZhy{}}\PY{o}{\PYZgt{}}\PY{n}{lk}\PY{p}{)}\PY{p}{;}
\PY{+w}{    }\PY{k}{goto}\PY{+w}{ }\PY{n}{retry}\PY{p}{;}
\PY{+w}{  }\PY{p}{\PYZcb{}}\PY{+w}{ }
\PY{+w}{  }\PY{n}{p}\PY{+w}{ }\PY{o}{=}\PY{+w}{ }\PY{n}{remove\PYZus{}min}\PY{p}{(}\PY{n}{h1}\PY{p}{)}\PY{p}{;}
\PY{+w}{  }\PY{n}{lock\PYZus{}release}\PY{p}{(}\PY{o}{\PYZam{}}\PY{n}{h1}\PY{o}{\PYZhy{}}\PY{o}{\PYZgt{}}\PY{n}{lk}\PY{p}{)}\PY{p}{;}
\PY{n+nl}{end}\PY{p}{:}
\PY{+w}{  }\PY{n}{lock\PYZus{}release}\PY{p}{(}\PY{o}{\PYZam{}}\PY{n}{h}\PY{o}{\PYZhy{}}\PY{o}{\PYZgt{}}\PY{n}{lk}\PY{p}{)}\PY{p}{;}
\PY{+w}{  }\PY{k}{return}\PY{+w}{ }\PY{n}{p}\PY{p}{;}
\PY{p}{\PYZcb{}}
\end{Verbatim}

\caption{The schedule function}
\label{fig:schedule}
\end{figure}






%\section{Implementation}\label{s:implementation}


In order to implement \beclass{} in Linux, we build on \schedidle{}, an existing
scheduling \textit{policy} in Linux. This is because \schedidle{} already has
some of the features we want from \beclass{}.

\subsection{\schedidle{} lends itself well to \beclass{}}

Scheduling \textit{classes} in Linux can have multiple \textit{policies}, and
\schedidle{} lives within the \normalclass{} class alongside the default policy
of the \normalclass{} class, which is \schednormal{}.\footnote{There is, very
confusingly, also an Idle scheduling \textit{class}, but that not accessible to
userspace and exists solely to manage the core's transition in and out of being
actually idle (ie running nothing).} The existing \schedidle{} policy is in many
ways not different from a low weight \schednormal{} process: both are kept on
the same runqueues as all the other \schednormal{} processes, and \schedidle{}
just has a predefined low weight of 3~\cite{weight-idleprio}.

One way that \schedidle{} is promising as a foundation for our implementation of
\beclass{} is that \schedidle{} was extended to be accessible via the \cgroups{}
API recently~\cite{lkml-idle-cgroup}: a whole groups' policy can be set to
\schedidle{} via the \cgroups{} interface. This means that building on
\schedidle{} allows us to get for free the ability to use the \cgroups{} API to
mark groups as BE. We thus have two ways to mark things as BE: we can mark
individual processes by setting their policy to \schedidle{}, or we can do so
for a whole group via the \cgroups{} API.

An additional benefit to using \schedidle{} as the foundation for our
implementation is that, after a push by Facebook, Linux developers already added
what is in effect the \entry{} check from the \beclass{}
design~\cite{fixing-idle-article}. In 2019, Linux added a check when a
\schednormal{} entity becomes newly runnable on a core already running something
in \schednormal{}. This new check looks for other cores that might be currently
running a \schedidle{} entity, and migrates the new entity there.

\subsection{\schedbe{} as an implementation of \beclass{}}

To implement \beclass{}, we modify \schedidle{} to add the \local{} and \exit{}
parts of the \beclass{} design. We call the resulting policy \schedbe{}. While
it is technically still a policy, it implements the desired behavior of
\beclass{}, and as a result behaves as if it was a scheduling class. Our
implementation is a patch to Linux version 6.14.2, which implements the
\beclass{} class by extending \schedidle{} to become \schedbe{}.

To enforce the \local{} part of the design, which calls for the local (ie single
runqueue) isolation of \schedbe{} processes, we ensure that the task chosen to
run from the runqueue is only \schedbe{} if everything else on the runqueue is
as well (\autoref{ss:implementation:local}). To enforce the \exit{} check, we
add a synchronization point to complete the global policy enforcement
(\autoref{ss:implementation:exit}).

\subsection{Enforcing the local policy}\label{ss:implementation:local}

In order to enforce that ruqueues only run \schedbe{} threads when there are no
runnable \schednormal{} ones, the patch interferes in two places. 

Because in existing Linux \schedidle{} and \schednormal{} share a runqueue, so
will the new \schedbe{} and \schednormal{}.\hmng{I mean I guess they don't have
to... would make migration and all that stuff roll your own though} This means
that the function that chooses the next task from the runqeueue will be
potentially looking at a mix of both. We add an initial check that establishes
whether there are any \schednormal{} threads on the runqueue, and skips all
\schedbe{} ones if that is the case. 

The second change is necessary because the first breaks an existing eligibility
mechanism. In order to maintain fairness, Linux currently accounts for the
difference between the fair share processes should have gotten and the time they
actually got, and stores that `lag'. Processes that have gotten more time than
they should (ie have negative lag), are marked as ineligible and not considered
when choosing what to run next. Since \schedbe{} threads are now potentially not
being run for a long time, there is a potential for deadlock: a \schednormal{}
task has been running for a while and accrued enough time that its lag is
negative and it is ineligible. However, if the only other thread is \schedbe{}
then we won't run that either because there is a runnable \schednormal{} task on
the runqueue. In order to avoid this situation, the patch removes the
eligibility criterion in choosing what to run next.


\subsection{Enforcing the global policy}\label{ss:implementation:exit}

Ensuring the \entry{} and \exit{} checks requires interposing on Linux's wakeup
and exit codepaths. Linux already special-cases on the wakeup path, although
only checks if the thread itself is marked as idle, and not if the group as a
whole is, and for \schedbe{} we check both.\hmng{this is not exactly true, but
in a kinda complicated way}

\schedbe{} adds a check on the exit path: if the thread chosen to run next is
\schedbe{}, then the patch tries to steal a queued \schednormal{} task from a
different core. Specifically, it picks the core with the max number of queued
but runnable \schednormal{} threads --- but only steals one, in order to not
overzealously steal. Linux does the same thing when a core would otherwise go
completely idle.
\section{Implementing \beclass{} in Linux}\label{s:implementation}

%-------------------------------------------------------------------------------
\section{Preliminary Results}
%-------------------------------------------------------------------------------



In order to understand the case for \sys{}, we ask the following questions: 
\begin{enumerate}
    \item How does job latency in \sys{} compare to schedulers without
    priorities on one hand, and theoreticlaly optimal schedulers with perfect
    information on the other?
    \item Does \sys{} plan for managing memory work?
\end{enumerate}


To explore these questions, we built a simulator in go\cite{TODO}, which
simulates different scheduling approaches.


\subsection{Experimental Setup}

In each version of the simulator, jobs arrive in an open loop at a constant
rate. The simulator attaches three main characteristics to each job it
generates: runtime, priority, and memory usage.\ \textit{Job runtime} is chosen
by sampling from randomly generated long tailed (in this case pareto)
distribution: the relative length of the tail ($\alpha$ value) remains constant,
and the minimum value ($x_m$) is chosen from a normal distribution. This
reflects the fact that different functions have different expected runtimes
(chosen from a normal distribution), and that actual job runtimes follow long
tailed distributions (so each pareto distribution that we sample represents the
expected runtime distribution of a given function).\ \textit{Job priority} is
chosen randomly, but weighted: the simulator uses a vaguely bimodal weighting
across priorties. The simulator has n different priority values, each assigned
to a fictitious price. Because functions are randomly assigned a priority,
runtime and priority are not correlated.\ \textit{Job memory usage} is chosen
randomly between 100MB and 10GB.
% from a bimodel normal distribution, with peaks at 1GB and at 8GB. Over their
% lifetime, jobs increase their memory usage from an initial amount (always 100MB)
% to their total usage.

When comparing two different simulated schedulers, they each are given an
identical workload and then each simulate running that workload.

The simulator makes some simplifying assumptions:
\begin{enumerate}
    \item functions are compute bound, and do not block for i/o
    \item communication latencies are not simulated
    \item amount of time it takes to page memory is not simulated
\end{enumerate}

We simulate running 100 machines, with 8 cores and 32GB RAM each, with 4
scheduler shards and a k-choices value of 3 when applicable.

\subsection{How do job latencies compare?}

\begin{figure}[t!]
    \centering
      \includegraphics[width=8cm]{img/hermod_xx_edf_latency.png}
      \caption{ add caption }
    \label{fig:hermod-xx-edf}
\end{figure}

In the end developers care about job latency, so it is important to understand
how well priorities do at reflecting and enforcing SLAs. On one hand, is
relevant to understand if we need priorities at all: is there a scheduler that
can, without having any access to information about which jobs are important,
still ensure that jobs perform well? On the other hand, it is helpful to compare
\sys{} to an ideal scheduler, in order to contextualize \sys{}'s performance.

To explore one side of this question, we look at how an existing state of the
art research scheduler that does not take any form of priority into account
performs. We simulate Hermod\cite{TODO}, a state-of-the-art serverless scheduler
built specifically for serverless. Hermod's design is the result of a
from-first-principles analysis of different scheduling paradigms. In accordance
with the paper's findings, we simulate least-loaded load balancing over machines
found using power-of-k-choices, combined with early binding and Processor
Sharing machine-level scheduling. Hermod does not use priorities in its design,
and as such the simulator ignores jobs' priority when simulating Hermod's
design.


On the other side, we want to simulate an ideal scheduler. Ideal here is with
respect to meeting jobs' SLAs, which requires defining the desired SLA. In order
to do this, we assign each job invocation a deadline, and allow the deadline to
be a function of the job's true runtime. We define the deadline as a function of
the runtime as well as the priority, as follows: deadline = runtime *
maxPrice/price (as if each process were weighted by its price). This ensures
that highest priority jobs' deadlines are simply their runtimes, and deadlines
get more and more slack with lower priorities. We then simulate an Earliest
Deadline First (EDF) scheduler over these deadlines, which is queuing
theoretically proven to be optimal in exactly the way we wanted: if it is
possible to create a schedule where all jobs meet their deadline, EDF will find
it\cite{TODO}.
% https://en.wikipedia.org/wiki/Earliest_deadline_first_scheduling

We compare the latencies observed in both of these settings with those that
running \sys{} produces. Because Hermod does not talk about dealing with memory
pressure, and to avoid an unfair comaprison with \sys{}'s paging, we set the
memory to be absurdly high for all three settings in this experiment. We also
turn off the use of the idle list in \sys{}, so as to be en par with Hermod in
placing load, and revert solely to k-choices.

A strong result for \sys{} would show that its performance is between the two,
and closer to the EDF side. Especially as load and utillization get high, we
expect that the differences betweeen the three approaches will become evident.
Figure~\ref{hermod-xx-edf} shows the results. As expected, \sys{} outperforms
Hermod once the utilization is above YYY, because Hermod spreads the slowdown
across all the jobs, whereas \sys{} can be more targeted.~\sys{} performs
comparably to EDF, until very high load. This is because EDF, because it has
knowledge of jobs' approximate runtime, stops running the longer low priority
jobs, and thus can finish the low priority jobs it does start more quickly.


\subsection{Does \sys{} plan for memory management work?}

\begin{figure}[t!]
    \centering
      \includegraphics[width=8cm]{img/memory_graphs.png}
      \caption{ add caption }
    \label{fig:memory-graphs}
\end{figure}

To answer this question, we look at how \sys{} distributes load, and whether the
amount that \sys{} needs to page memory is realistic. We now run \sys{} in a
setting of limited memory, and track the memory utilization of different
machines, as well as how much and what they need to page. A good result would
show a tigh spread of memory utilization, that machines only start paging once
memory utilization is high, and that the amount of paging being done is also
equally spread across machines. Figure~\ref{fig:memory-graphs} shows the
results.



\section{Related work}

Some work focuses on isolating real-time applications in
Linux~\cite{rt-in-linux, state-rt-linux}. Wasted Cores~\cite{wasted-cores}
addresses the idle behavior of Linux. 

Linux's \schedidle{} policy addresses the observed weight inversion, but is
still inadequate: running the microbenchmark using \schedidle{} leads to an
increase in the mean latency from 6ms to 8ms.

Other systems work around the kernel scheduler, by running primarily in
userspace~\cite{perfiso,caladan,skyloft}.
%-------------------------------------------------------------------------------
\section{Conclusion}
%-------------------------------------------------------------------------------


Serverless was and is a great option for developers whose load varies and
providers who don't want to keep resources idle for processes that reserved
them. However, the reality of serverless today is that functions experience a
variance in latencies that is not tolerable for latency sensitive workloads. 

The common approach to this is to work on cold starts. This paper asks what
comes next; once cold starts in the single digit ms range (which we are starting
to see in research schedulers) are commercially available, have we figured out a
system that can run serverless in its ideal form?

We show that this is not the case, and that once more latency senstive functions
are able to run alongside the usual map reduce and image resize functions, we
will need some way of prioritising which functions are latency sensitive in
order to keep their latencies acceptable.

We propose a new scheduler, \sys{}, that introduces \emph{\priceclass{}es}.
Developers assign each function they want to run to a \priceclass{}, which
encodes a priority that \sys{} then enforces at invocation, both in placing the
function and on the machine level by running priority scheduling. We show that
\sys{} is able to enforce priorities and keep high \class{} functions latencies
stable even under high load.

%-------------------------------------------------------------------------------

\bibliography{paper}{}
\bibliographystyle{plainnat}

%%%%%%%%%%%%%%%%%%%%%%%%%%%%%%%%%%%%%%%%%%%%%%%%%%%%%%%%%%%%%%%%%%%%%%%%%%%%%%%%
\end{document}
%%%%%%%%%%%%%%%%%%%%%%%%%%%%%%%%%%%%%%%%%%%%%%%%%%%%%%%%%%%%%%%%%%%%%%%%%%%%%%%%

%%  LocalWords:  endnotes includegraphics fread ptr nobj noindent
%%  LocalWords:  pdflatex acks

\section{Solution}\label{s:maybe-solution}

We implement a new category for processes to be in, which we call \schedbe{},
which exists in relation to the default scheduling policy, \schednormal{}. The
goal is to enforce the maxim that no \schedbe{} userspace thread is ever running
if a \schednormal{} task is waiting for a core. In order to do so, the scheduler
needs to enforce the priorities in three different places:
\begin{enumerate}
    \item in picking the next entity from the runqueue on each core, it needs to
ensure that no \schedbe{} entity will be chosen if there is a runnable
\schednormal{} entity,
    \item it needs to try to steal queued \schednormal{} entities from other cores
before running a \schedbe{} entity, if one was chosen,
    \item when waking up a \schednormal{} entities on a core already running a
    \schednormal{} entity, it needs to look for other cores running \schedbe{}
    entities to go interrupt.
\end{enumerate}

These three pieces together create the desired strict and global priority. The
first ensures the property locally on each core. The second and third enforce it
globally: given the local priority, the scheduler needs to sychronize at two
points in order to enforce it globally. Once on \textit{entry}, when a new
high-priority thread wakes up on a core already running something high-priority,
and once on \textit{exit}, when starting to run a low priority thread. This way,
if a core is currently running a \schedbe{} thread $t$, the scheduler knows that
there are no queued and waiting \schednormal{} threads, because if there were
ones before it started running $t$ the scheduler would have stolen it, and if a
new \schednormal{} thread wakes up while $t$ is running, before enqueueing it
the core where it wakes up will look to interrupt cores running \schedbe{}.


\subsection{Implementing \schedbe{} in Linux}

We implement \schedbe{} in Linux, as an extension of the existing \schedidle{}
scheduling policy. We use \schedidle{} to refer to the existing implementation,
and \schedbe{} to the categorically separate priority for BE and its
implementation as an edited \schedidle{}. We compare against it as a baseline in
\autoref{ss:schedidle}. 

\schedidle{} is in many ways not too different from a low weight \schednormal{}
process: both are kept on the same runqueues as all the other \schednormal{}
processes, and \schedidle{} just has a predefined low weight of
3~\cite{TODO}.\footnote{There is, confusingly, also an Idle scheduling
\textit{class}, but that not accessible to userspace and exists solely to manage
the core's transition in and out of being actually idle (ie running nothing).}

However, Linux has over the years added more special-casing to \schedidle{}. In
2019, they added a check when a \schednormal{} entity becomes newly runnable on
a core already running something in \schednormal{}. This new check will look for
other cores that might be currently running a \schedidle{} entity, and migrates
the new entity there. This represents the \textit{entry check}.

The other special thing Linux added is that \schedidle{} was extended to have
cgroup support recently\cite{TODO}: a whole groups' policy can be set to
\schedidle{} via the \cgroups{} interface. This also makes it additionally
promising to be used as a basis for \schedbe{} because it can fit well with the
existing infrastructure, which uses the \cgroups{} api.

So, the actual patch to Linux that creates \schedbe{} builds on \schedidle{},
and adds the missing components from the list: the patch enforces proper policy
for \schedbe{} processes locally by ensuring that the task chosen to run from
the runqueue is only \schedidle{} if everything else on the runqueue is as well,
and it adds an \textit{exit} synchronization point to complete the global policy
enforcement. In doing so, the patch creates \schedbe{}.

\subsection{Enforcing the local policy}

In order to enforce that ruqueues only run \schedbe{} threads when there are no
runnable \schednormal{} ones, the patch interferes in two places. 

Because in existing Linux \schedidle{} and \schednormal{} share a runqueue, so
will the new \schedbe{} and \schednormal{}. This means that the actual function
that chooses the next task from the runqeueue will be potentially looking at a
mix of both. For \schedbe{}, we add an initial check that establishes whether
there are any \schednormal{} threads on the runqueue, and skips all \schedbe{}
ones if that is the case. 

The second change is necessary because the first throws the fair share
eligibility mechanism out of whack. In order to maintain fairness, Linux
currently accounts for the difference between the fair share processes should
have gotten and the time they actually got, and stores that `lag'. Even low
weight processes accrue lag over time, albeit more slowly. Since \schedbe{}
threads are now potentially not being run for a long time, there is a potential
for deadlock: a \schednormal{} task has been running for a while and accrued
enough time that its lag is negative and it is ineligible. However, if the only
other thread is \schedbe{} then we won't run that either. In order to avoid this
situation, the patch removes the eligibility criterion in choosing what to run
next.


\subsection{Enforcing the global policy}

Ensuring the exit and entry path requires interposing on both. The existing code
already special-cases on the wakeup path, although only checks if the thread
itself is marked as idle, and not if the group as a whole is, and for \schedbe{}
we check both.\hmng{this is not exactly true, but in a kinda complicated way. Do
we even want to say this?}

\schedbe{} also adds a check on the exit: if the thread chosen to run next is
\schedbe{}, then the patch tries to steal a queued \schednormal{} task from a
different core. Specifically, it picks the core with the max number of queued
but runnable \schednormal{} threads, but only steals one. This is in order to
not overzealously steal, given that it is the load blanacers job to ensure that
in general load is well balanced.

%-------------------------------------------------------------------------------
\section{Conclusion}
%-------------------------------------------------------------------------------

This work shows that current isolation systems like Kubernetes are unable to
effectively performance isolate between LC and BE workloads for workloads that
are CPU-intensive. We trace this problem down to Linux's \cgroups{}, and show
that because Linux uses per-core runqueues, its weight-based interface for
isolation is not enforced across cores. Using a weight-based interface for
isolating LC and BE has other problems as well: it makes it hard to enforce a
split across cores, and it interacts poorly with the large scheduling quantum
that Linux uses.

Instead, we propose an API that categorically separates BEs from LCs, and argue
that the scheduler should implement a strict and categorical isolation between
the two. This requires less synchronization than a weight-based isolation
approach, and ensures that no BE is ever running when an LC is queued. During
high load this requires `parking', which enforces that BEs are immediately
runnable once the load goes down, with minimal interference for LCs in the
meantime while the load is high.

We implement this categorical split in Linu, and show that these changes allow
\cgroups{} itself as well as higher level applications that build on \cgroups{},
like Kubernetes, to do a better job of isolating LC from BE workloads.


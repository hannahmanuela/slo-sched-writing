%-------------------------------------------------------------------------------
\section{Conclusion}
%-------------------------------------------------------------------------------

This work shows that current isolation systems like Kubernetes are unable to
effectively isolate the latencies of LC applications from BE workloads for tasks
that are CPU-intensive. We trace this problem down to Linux's \cgroups{}, and
show that because Linux uses per-core runqueues, its weight-based interface is
not enforced across cores. Using a weight-based interface to enforce LC
processes' reservations on cores also running BE processes has other problems as
well: it makes it hard to enforce a split across cores, and it interacts poorly
with the large scheduling quantum that Linux uses.

Instead, we propose an API that separates best effort workloads from critical
ones with reservations by introducing the \beclass{} priority class. \beclass{}
requires fewer cross-core interactions than a weight-based approach, and ensures
that no BE is ever running when an LC is queued. During high load this requires
`parking', which enforces that BEs are immediately runnable once the load goes
down, with minimal interference for LCs in the meantime while the load is high.

We implement this strict priority in Linux, and show that the resulting
\schedbe{} allows \cgroups{} itself, as well as higher level applications that
build on \cgroups{} like Kubernetes, to do a better job of ensuring LC
processes' access to their reserved cores while running BE workloads
opportunistically. Using \schedbe{} rather than the standard \cgroups{} weight
interface decreases the impact starting a BE process has on LC latencies from
>2x to 0, and under high load a parked \schedbe{} process running in a
Kubernetes pod can resume execution normally after multiple minutes of no user
space CPU time.


\section{Related Work}

Much previous has targeted improving the isolation of latency critical
applications' latencies from other tasks in the system.

\textbf{Linux-related work} Some existing work focuses on Linux. Many focus on
real-time applications~\cite{rt-in-linux, state-rt-linux}, which have stricter
latency requirements than the microservice applications targeted in this paper,
and have more predictable execution behavior. As a result, these systems build
on the real time scheduling classes.\hmng{cld use better cites}

A Decade of Wasted Cores~\cite{wasted-cores} focuses on the idle behavior of the
Linux scheduler. It does not find fault with the scheduling interface, but
focuses on the internal migration and stealing behavior of cores in order to
improve performance and utilization. By contrast, this work changes the way
developers express what is latency critical and what is best effort to the
kernel.

\textbf{Other machine schedulers} Other systems work around the kernel
scheduler, by making policy decisions, handling cross-core communication, and
enforcing interrupts, in userspace.

Systems like PerfIso~\cite{perfiso} reactively adjust the amount of cores that
are allocated to different workloads, but still using kernel mechanisms to
enforce the split. Unlike \cgroups{} weights, these splits are are at the
fidelity of whole CPUs, and require unused `buffer' cores for the LC tasks in
case their load spikes.

Caladan~\cite{caladan} also reactively adjusts core assignment, and does so to
mitigate I/O as well as CPU interference. Unlike this work, Caladan replaces the
kernel scheduler with a userspace centralized scheduler, and targets a
preemption granularity at the micro-second scale. Skyloft~\cite{skyloft}
similarly uses a userspace scheduler to achieve micro-second scale preemption,
and supports LC as well as BE workloads.

Other systems like Syrup~\cite{syrup} or ghOSt~\cite{ghost} allow users to
create their own scheduling policies, whereas this work focuses on an existing
popular scheduling approach and implements it more efficiently in the kernel. 




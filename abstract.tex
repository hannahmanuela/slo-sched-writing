\begin{abstract}
    The promise of serverless currently remains an elusive and extremely
    attractive paradigm: get what you need as you need it, but only pay for what
    you use. 
    
    For example, building an elastic web server on top of a serverless
    infrastructure is not something that can be done easily today, but is in
    principle a good candidate for the serverless setting: its load is
    unpredictable and bursty, and it does not require managing its data locally. 
    
    Many challenges remain in making the ideal of serverless a reality; \sys{}
    tackles scheduling: in a world where a large and heterogeneous set of jobs
    is being run on as serverless functions, there has to be a way to
    differentiate between functions’ requirements and urgency. 
    
    \sys{} is a distributed scheduler that allows developers to express priorities
    and enforces them. Developers express priorities to \sys{} via assigning
    functions to fixed dollar amounts per unit of compute, and cap the overall
    usage by specifying a monthly budget. Memory costs per unit time used are
    the same across all priorities, and developers specify a maximum amount of
    memory for each function.\hmng{ Do I need the memory blurb? I added it because I
    mention memory below } \sys{} places incoming jobs on machines that have memory
    available, and implements a machine scheduler that enforces priorities. 
    
    \hmng{ TODO impl/eval blurb }

\end{abstract}

\section{Results}

\begin{figure}[t]
    \centering
    \includegraphics[width=\columnwidth]{graphs/kubernetes-schedbe.png}
    \caption{ Kubernetes using \beclass{} honors reservations, unlike in
    \autoref{fig:kubernetes-unedited} }\label{fig:kubernetes-schedbe}
\end{figure}
% Kubernetes using \beclass{} honors reservations, unlike in
%     \autoref{fig:kubernetes-unedited}

We re-run the Kubernetes application using \beclass{}.
\autoref{fig:kubernetes-schedbe} shows that the baseline mean latency of the LC
server stays around 6.5ms after starting the the BE. Looking at the trace shows
the BE runs only in gaps where cores would otherwise be idle.

\begin{figure}[t]
    \centering
    \includegraphics[width=\columnwidth]{graphs/overload-schedbe.png}
    \caption{ \beclass{} starves BE user-space threads, leaving all CPU-time to
    the LC}\label{fig:overload-schedbe}
\end{figure}
% \beclass{} starves BE user-space threads, leaving all CPU-time to
%     the LC

\autoref{fig:overload-schedbe} shows parking enables the server to keep its
reservation even under sustained 100\% load. We run the same client load as
\autoref{fig:overload-rt}, with \beclass{} parking the BE. Notice that the BE
does not make progress until the end, when the server is done processing.

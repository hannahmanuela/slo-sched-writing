% From mitthesis package
% Version: 1.01, 2023/06/19
% Documentation: https://ctan.org/pkg/mitthesis
%
% The abstract environment creates all the required headers and footnote. 
% You only need to add the text of the abstract itself.
%
% Approximately 500 words or less; try not to use formulas or special characters
% If you don't want an initial indentation, do \noindent at the start of the abstract

Linux is an operating system that underlies most of modern computing
infrastructure. As such, it has conflicting requirements of being both general
purpose and performant. This thesis identifies a limitation in Linux's interface
for a use case that is both generic enough and important enough to deserve first
class support in the kernel: the separation of latency critical (LC) workloads
from best effort (BE) ones. The current interface for separating these two, used
by popular containerization software such as Kubernetes and AFaaS, is based on
weights and is poorly enforced. This thesis explores making the LC/BE split
explicit by putting best effort work in a separate scheduling class. We show
that doing so enables Linux to enforce the LC/BE split across cores with
acceptable overheads, and that this stabilizes the performance of LC workload in
the presence of BE.